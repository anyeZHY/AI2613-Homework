% ! Tex program = xelatex
\documentclass{article}
% \PassOptionsToPackage{quiet}{fontspec}% (or try silent)
% 中文
% \usepackage[UTF8]{ctex}

% For more choices
% %! Tex program = xelatex
% \documentclass{article}
%中文
%\usepackage[UTF8]{ctex}
%数学公式
\usepackage{amsmath,amssymb}
%\usepackage{ntheorem}
% \usepackage[framemethod=TikZ]{mdframed}
\usepackage{amsthm}
%边界
\usepackage[letterpaper,top=2.5cm,bottom=2.5cm,left=3cm,right=3cm,marginparwidth=1.75cm]{geometry}%table package
%Table
\usepackage{multirow,booktabs}
\usepackage{makecell}
%字体颜色
\usepackage{color}
\usepackage[dvipsnames]{xcolor}  % 更全的色系
%代码
\usepackage[OT1]{fontenc}
% MATLAB 代码风格
%\usepackage[framed,numbered,autolinebreaks,useliterate]{/Users/anye_zhenhaoyu/Desktop/Latex/mcode}
\usepackage{listings}
\usepackage{algorithm}
\usepackage{algorithmic}
\usepackage{pythonhighlight} % Python
%插图
\usepackage{graphicx}
%改变item格式
\usepackage{enumerate}
%物理
\usepackage{physics}
%extra arrows
\usepackage{extarrows}
% caption(居中指令)
%\usepackage[justification=centering]{caption}
\usepackage{caption}
% htpb
\usepackage{stfloats}
% pdf 拼接
\usepackage{pdfpages}
% 超链接url
\usepackage{url}
% \usepackage{tikz}
\usepackage{pgfplots}
\pgfplotsset{compat=newest}
\usepackage[colorlinks=true, allcolors=blue]{hyperref}
\usepackage{setspace}

% --------------definations-------------- %
\def\*#1{\boldsymbol{#1}}
\def\+#1{\mathcal{#1}} 
\def\-#1{\mathrm{#1}}
\def\=#1{\mathbb{#1}}
% Domains
\def\RR{\mathbb{R}}
\def\CC{\mathbb{C}}
\def\NN{\mathbb{N}}
\def\ZZ{\mathbb{Z}}
% Newcommand
\newcommand{\inner}[2]{\langle #1,#2\rangle} 
\newcommand{\numP}{\#\mathbf{P}} 
\renewcommand{\P}{\mathbf{P}}
\newcommand{\Var}[2][]{\mathbf{Var}_{#1}\left[#2\right]}
\newcommand{\E}[2][]{\mathbf{E}_{#1}\left[#2\right]}
\renewcommand{\emptyset}{\varnothing}
\newcommand{\ol}{\overline}
\newcommand{\argmin}{\mathop{\arg\min}}
\newcommand{\argmax}{\mathop{\arg\max}}
\renewcommand{\abs}[1]{\qty|#1|}
\newcommand{\defeq}{\triangleq} % triangle over =
\def\deq{\xlongequal{def}} % 'def' over =
\def\LHS{\text{LHS}}
\def\RHS{\text{RHS}}
\def\angbr#1{\langle#1\rangle} % <x>
\newcommand\set[1]{\qty{#1}}

\def\Esolve{\textcolor{blue}{Solve: }}
\def\Eproof{\textcolor{blue}{Proof: }}
\def\case#1{\textcolor{blue}{Case \uppercase\expandafter{\romannumeral#1}: }}
\def\card#1{\mbox{Card}(#1)}

% \newmdtheoremenv{lemma}{Lemma}
% \newmdtheoremenv{theorem}{\textcolor{red}{Theorem}}
% \newmdtheoremenv{defi}{\textcolor{blue}{Definition}}
\newtheorem{lemma}{Lemma}
\newtheorem{prp}{Proposition}
\newtheorem{thm}{Theorem}
\newtheorem{defi}{Defination}

\newenvironment{md}{\begin{mdframed}}{\end{mdframed}}
\usepackage{fancyhdr}
\pagestyle{fancy}
% \fancypagestyle{mainFancy}{
%     \fancyhf{}
%     \renewcommand\headrulewidth{.5pt}       % 页眉横线
%     \renewcommand\footrulewidth{0pt}
%     \fancyhead[OC]{\fzkai{\leftmark}}       % 页眉章标题
%     \fancyhead[EC]{\fzkai{\@title}}         % 页眉文章题目
% 	\lhead{\fzkai{author}}
%     \fancyhead[OR,EL]{\thepage}                 % 页眉编号
%     \fancyfoot[r]{\thumb} % 将拇指放到没有被使用的页眉或页脚处
% }
\fancyhf{}
\fancyhead[L]{\slshape{Haoyu Zhen}}
\fancyfoot[C]{\thepage}
\fancyhead[R]{\slshape{Available at \href{https://github.com/anyeZHY/AI2613-Homework}{https://github.com/anyeZHY/AI2613-Homework}}}

\graphicspath{{figures/}}

% \begin{document}
% \title{<++>}
\author{Haoyu Zhen}
% \maketitle
\setlength{\parindent}{0pt}
\setstretch{1.3}
% \end{document}


% \input{/Users/anye_zhenhaoyu/Desktop/ln_preamble.tex}

% %! Tex program = xelatex
% \documentclass{article}
%中文
%\usepackage[UTF8]{ctex}
%数学公式
\usepackage{amsmath,amssymb}
%\usepackage{ntheorem}
% \usepackage[framemethod=TikZ]{mdframed}
\usepackage{amsthm}
%边界
\usepackage[letterpaper,top=2.5cm,bottom=2.5cm,left=3cm,right=3cm,marginparwidth=1.75cm]{geometry}%table package
%Table
\usepackage{multirow,booktabs}
\usepackage{makecell}
%字体颜色
\usepackage{color}
\usepackage[dvipsnames]{xcolor}  % 更全的色系
%代码
\usepackage[OT1]{fontenc}
% MATLAB 代码风格
%\usepackage[framed,numbered,autolinebreaks,useliterate]{/Users/anye_zhenhaoyu/Desktop/Latex/mcode}
\usepackage{listings}
\usepackage{algorithm}
\usepackage{algorithmic}
\usepackage{pythonhighlight} % Python
%插图
\usepackage{graphicx}
%改变item格式
\usepackage{enumerate}
%物理
\usepackage{physics}
%extra arrows
\usepackage{extarrows}
% caption(居中指令)
%\usepackage[justification=centering]{caption}
\usepackage{caption}
% htpb
\usepackage{stfloats}
% pdf 拼接
\usepackage{pdfpages}
% 超链接url
\usepackage{url}
% \usepackage{tikz}
\usepackage{pgfplots}
\pgfplotsset{compat=newest}
\usepackage[colorlinks=true, allcolors=blue]{hyperref}
\usepackage{setspace}

% --------------definations-------------- %
\def\*#1{\boldsymbol{#1}}
\def\+#1{\mathcal{#1}} 
\def\-#1{\mathrm{#1}}
\def\=#1{\mathbb{#1}}
% Domains
\def\RR{\mathbb{R}}
\def\CC{\mathbb{C}}
\def\NN{\mathbb{N}}
\def\ZZ{\mathbb{Z}}
% Newcommand
\newcommand{\inner}[2]{\langle #1,#2\rangle} 
\newcommand{\numP}{\#\mathbf{P}} 
\renewcommand{\P}{\mathbf{P}}
\newcommand{\Var}[2][]{\mathbf{Var}_{#1}\left[#2\right]}
\newcommand{\E}[2][]{\mathbf{E}_{#1}\left[#2\right]}
\renewcommand{\emptyset}{\varnothing}
\newcommand{\ol}{\overline}
\newcommand{\argmin}{\mathop{\arg\min}}
\newcommand{\argmax}{\mathop{\arg\max}}
\renewcommand{\abs}[1]{\qty|#1|}
\newcommand{\defeq}{\triangleq} % triangle over =
\def\deq{\xlongequal{def}} % 'def' over =
\def\LHS{\text{LHS}}
\def\RHS{\text{RHS}}
\def\angbr#1{\langle#1\rangle} % <x>
\newcommand\set[1]{\qty{#1}}

\def\Esolve{\textcolor{blue}{Solve: }}
\def\Eproof{\textcolor{blue}{Proof: }}
\def\case#1{\textcolor{blue}{Case \uppercase\expandafter{\romannumeral#1}: }}
\def\card#1{\mbox{Card}(#1)}

% \newmdtheoremenv{lemma}{Lemma}
% \newmdtheoremenv{theorem}{\textcolor{red}{Theorem}}
% \newmdtheoremenv{defi}{\textcolor{blue}{Definition}}
\newtheorem{lemma}{Lemma}
\newtheorem{prp}{Proposition}
\newtheorem{thm}{Theorem}
\newtheorem{defi}{Defination}

\newenvironment{md}{\begin{mdframed}}{\end{mdframed}}
\usepackage{fancyhdr}
\pagestyle{fancy}
% \fancypagestyle{mainFancy}{
%     \fancyhf{}
%     \renewcommand\headrulewidth{.5pt}       % 页眉横线
%     \renewcommand\footrulewidth{0pt}
%     \fancyhead[OC]{\fzkai{\leftmark}}       % 页眉章标题
%     \fancyhead[EC]{\fzkai{\@title}}         % 页眉文章题目
% 	\lhead{\fzkai{author}}
%     \fancyhead[OR,EL]{\thepage}                 % 页眉编号
%     \fancyfoot[r]{\thumb} % 将拇指放到没有被使用的页眉或页脚处
% }
\fancyhf{}
\fancyhead[L]{\slshape{Haoyu Zhen}}
\fancyfoot[C]{\thepage}
\fancyhead[R]{\slshape{Available at \href{https://github.com/anyeZHY/AI2613-Homework}{https://github.com/anyeZHY/AI2613-Homework}}}

\graphicspath{{figures/}}

% \begin{document}
% \title{<++>}
\author{Haoyu Zhen}
% \maketitle
\setlength{\parindent}{0pt}
\setstretch{1.3}
% \end{document}


% \input{/path/ln_preamble.tex}

% On my MAC's Desktop
%! Tex program = xelatex
% \documentclass{article}
%中文
%\usepackage[UTF8]{ctex}
%数学公式
\usepackage{amsmath,amssymb}
%\usepackage{ntheorem}
% \usepackage[framemethod=TikZ]{mdframed}
\usepackage{amsthm}
%边界
\usepackage[letterpaper,top=2.5cm,bottom=2.5cm,left=3cm,right=3cm,marginparwidth=1.75cm]{geometry}%table package
%Table
\usepackage{multirow,booktabs}
\usepackage{makecell}
%字体颜色
\usepackage{color}
\usepackage[dvipsnames]{xcolor}  % 更全的色系
%代码
\usepackage[OT1]{fontenc}
% MATLAB 代码风格
%\usepackage[framed,numbered,autolinebreaks,useliterate]{/Users/anye_zhenhaoyu/Desktop/Latex/mcode}
\usepackage{listings}
\usepackage{algorithm}
\usepackage{algorithmic}
\usepackage{pythonhighlight} % Python
%插图
\usepackage{graphicx}
%改变item格式
\usepackage{enumerate}
%物理
\usepackage{physics}
%extra arrows
\usepackage{extarrows}
% caption(居中指令)
%\usepackage[justification=centering]{caption}
\usepackage{caption}
% htpb
\usepackage{stfloats}
% pdf 拼接
\usepackage{pdfpages}
% 超链接url
\usepackage{url}
% \usepackage{tikz}
\usepackage{pgfplots}
\pgfplotsset{compat=newest}
\usepackage[colorlinks=true, allcolors=blue]{hyperref}
\usepackage{setspace}

% --------------definations-------------- %
\def\*#1{\boldsymbol{#1}}
\def\+#1{\mathcal{#1}} 
\def\-#1{\mathrm{#1}}
\def\=#1{\mathbb{#1}}
% Domains
\def\RR{\mathbb{R}}
\def\CC{\mathbb{C}}
\def\NN{\mathbb{N}}
\def\ZZ{\mathbb{Z}}
% Newcommand
\newcommand{\inner}[2]{\langle #1,#2\rangle} 
\newcommand{\numP}{\#\mathbf{P}} 
\renewcommand{\P}{\mathbf{P}}
\newcommand{\Var}[2][]{\mathbf{Var}_{#1}\left[#2\right]}
\newcommand{\E}[2][]{\mathbf{E}_{#1}\left[#2\right]}
\renewcommand{\emptyset}{\varnothing}
\newcommand{\ol}{\overline}
\newcommand{\argmin}{\mathop{\arg\min}}
\newcommand{\argmax}{\mathop{\arg\max}}
\renewcommand{\abs}[1]{\qty|#1|}
\newcommand{\defeq}{\triangleq} % triangle over =
\def\deq{\xlongequal{def}} % 'def' over =
\def\LHS{\text{LHS}}
\def\RHS{\text{RHS}}
\def\angbr#1{\langle#1\rangle} % <x>
\newcommand\set[1]{\qty{#1}}

\def\Esolve{\textcolor{blue}{Solve: }}
\def\Eproof{\textcolor{blue}{Proof: }}
\def\case#1{\textcolor{blue}{Case \uppercase\expandafter{\romannumeral#1}: }}
\def\card#1{\mbox{Card}(#1)}

% \newmdtheoremenv{lemma}{Lemma}
% \newmdtheoremenv{theorem}{\textcolor{red}{Theorem}}
% \newmdtheoremenv{defi}{\textcolor{blue}{Definition}}
\newtheorem{lemma}{Lemma}
\newtheorem{prp}{Proposition}
\newtheorem{thm}{Theorem}
\newtheorem{defi}{Defination}

\newenvironment{md}{\begin{mdframed}}{\end{mdframed}}
\usepackage{fancyhdr}
\pagestyle{fancy}
% \fancypagestyle{mainFancy}{
%     \fancyhf{}
%     \renewcommand\headrulewidth{.5pt}       % 页眉横线
%     \renewcommand\footrulewidth{0pt}
%     \fancyhead[OC]{\fzkai{\leftmark}}       % 页眉章标题
%     \fancyhead[EC]{\fzkai{\@title}}         % 页眉文章题目
% 	\lhead{\fzkai{author}}
%     \fancyhead[OR,EL]{\thepage}                 % 页眉编号
%     \fancyfoot[r]{\thumb} % 将拇指放到没有被使用的页眉或页脚处
% }
\fancyhf{}
\fancyhead[L]{\slshape{Haoyu Zhen}}
\fancyfoot[C]{\thepage}
\fancyhead[R]{\slshape{Available at \href{https://github.com/anyeZHY/AI2613-Homework}{https://github.com/anyeZHY/AI2613-Homework}}}

\graphicspath{{figures/}}

% \begin{document}
% \title{<++>}
\author{Haoyu Zhen}
% \maketitle
\setlength{\parindent}{0pt}
\setstretch{1.3}
% \end{document}



\graphicspath{{figures/}}

\begin{document}
% \tableofcontents
\title{Homework 5}
\maketitle

\section*{Problem 1}

By defination, $n\sim\-{Pois}(s\lambda)$. Thus
\[
	\Pr(\text{achieve his goal})=\Pr(n=1)=s\lambda\exp(-s\lambda)
.\] 
Then
\[
	s_{\-{opt}}=\argmax_s\Big[s\lambda\exp(-s\lambda)\Big]=\frac{1}{\lambda}
.\]


\section*{Problem 2}
\subsection*{(a)}
Apparently
\[
    \begin{aligned}
		&\Pr[X=\lambda+k]\ge \Pr[X=\lambda-k-1]
		\\[4pt]\iff&
		\flatfrac{\lambda^{\lambda+k}}{(\lambda+k)!}\ge
		\flatfrac{\lambda^{\lambda-k-1}}{(\lambda-k-1)!}
		\\\iff&
		\lambda^{2k+1}\ge \prod^k_{i=-k}(\lambda+i)
    \end{aligned}
\] 
And the last line holds by the fact that $(\lambda-i)(\lambda+i)\le \lambda^2$.
Then we have 
\[
	2\Pr[X\ge \lambda]\ge \Pr[X\ge \lambda]+\Pr[X<\lambda]=1
\]
which entails that $\Pr[X\ge \lambda]\ge \flatfrac{1}{2}$
\subsection*{(b)}
Let $\mathbb{E}_m[X]$ denotes the expectation of  $X$ with respect to $m$.
Intuitively and inspired by Cor.4,
\[
    \begin{aligned}
		\mathbb{E}_m[f(Y_1,\cdots,Y_n)]
		&=
		\sum^{\infty}_{k=0} 
		\mathbb{E}\qty[f(Y_1,\cdots,Y_n)|\sum^n_{i}Y_i=k]\Pr[\sum^n_{i}Y_i=k]
	    \\&=
		\sum^{\infty}_{k=0}
		\mathbb{E}_k\qty[f(X_1,\cdots,X_n)]\Pr[\sum^n_{i}Y_i=k]
		\\&\ge
		\sum^{\infty}_{k=m}
		\mathbb{E}_k\qty[f(X_1,\cdots,X_n)]\Pr[\sum^n_{i}Y_i=k]
		\\&\ge 
		\mathbb{E}_m\qty[f(X_1,\cdots,X_n)]\sum^{\infty}_{k=m}\Pr[\sum^n_{i}Y_i=k]
		\\&\ge 
		\frac{1}{2}\mathbb{E}_m\qty[f(X_1,\cdots,X_n)]
    \end{aligned}
\] 
where the second inequality holds by $E_m[f(X_1,\cdots,X_n)]$ is monotonically increasing in $m$ and the third inequlity holds by (a).

\subsection*{(c)}
Firstly we define some useful symbols.
Let $X_i$ be the number of students whose birthday is on day  $i$ ($i\in[365]$). 
Then $f(X_1,\cdots,X_m)\defeq\mathbbm{1}(\exists i\in[m], X_i\ge 4)$ where $\mathbbm{1}(\cdot)$ is an indicator.
Hence we could deduce that $\Pr[f(X_1,\cdots,X_m)=1]=\mathbb{E}[f(X_1,\cdots,X_m)]$.
And obviously, $\Pr[f(X_1,\cdots,X_m)=1]$ is monotonically increasing in $n$ (the advent of a new student will increase the probability absolutely).
\[
	\Pr\Big[\exists i\in[m],X_i\ge 4\Big]
	\le 
	2\Pr\Big[\exists i\in[m],Y_i\ge 4\Big]=2\qty(1-\Pr[Y_1<4]^{365})
.\] 
Trivilly,
\[
	\Pr[Y_1<4]=\qty(1+\lambda+\frac{\lambda^2}{2}+\frac{\lambda^3}{6})\exp(-\lambda)
.\] 
When $\lambda=\flatfrac{50}{365}$
\[
    \Pr\Big[\exists i\in[m],X_i\ge 4\Big]
	\le 9.58\times 10^{-3}\le 1\%
.\] 
\end{document}

