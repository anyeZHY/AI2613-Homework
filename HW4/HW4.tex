% ! Tex program = xelatex
\documentclass{article}
% \PassOptionsToPackage{quiet}{fontspec}% (or try silent)
% 中文
% \usepackage[UTF8]{ctex}

% For more choices
% %! Tex program = xelatex
% \documentclass{article}
%中文
%\usepackage[UTF8]{ctex}
%数学公式
\usepackage{amsmath,amssymb}
%\usepackage{ntheorem}
% \usepackage[framemethod=TikZ]{mdframed}
\usepackage{amsthm}
%边界
\usepackage[letterpaper,top=2.5cm,bottom=2.5cm,left=3cm,right=3cm,marginparwidth=1.75cm]{geometry}%table package
%Table
\usepackage{multirow,booktabs}
\usepackage{makecell}
%字体颜色
\usepackage{color}
\usepackage[dvipsnames]{xcolor}  % 更全的色系
%代码
\usepackage[OT1]{fontenc}
% MATLAB 代码风格
%\usepackage[framed,numbered,autolinebreaks,useliterate]{/Users/anye_zhenhaoyu/Desktop/Latex/mcode}
\usepackage{listings}
\usepackage{algorithm}
\usepackage{algorithmic}
\usepackage{pythonhighlight} % Python
%插图
\usepackage{graphicx}
%改变item格式
\usepackage{enumerate}
%物理
\usepackage{physics}
%extra arrows
\usepackage{extarrows}
% caption(居中指令)
%\usepackage[justification=centering]{caption}
\usepackage{caption}
% htpb
\usepackage{stfloats}
% pdf 拼接
\usepackage{pdfpages}
% 超链接url
\usepackage{url}
% \usepackage{tikz}
\usepackage{pgfplots}
\pgfplotsset{compat=newest}
\usepackage[colorlinks=true, allcolors=blue]{hyperref}
\usepackage{setspace}

% --------------definations-------------- %
\def\*#1{\boldsymbol{#1}}
\def\+#1{\mathcal{#1}} 
\def\-#1{\bar{#1}}
% Domains
\def\expect{\mathbb{E}}
\def\RR{\mathbb{R}}
\def\CC{\mathbb{C}}
\def\NN{\mathbb{N}}
\def\ZZ{\mathbb{Z}}
% Newcommand
\newcommand{\inner}[2]{\langle #1,#2\rangle} 
\newcommand{\numP}{\#\mathbf{P}} 
\renewcommand{\P}{\mathbf{P}}
\newcommand{\Var}[2][]{\mathbf{Var}_{#1}\left[#2\right]}
\newcommand{\E}[2][]{\mathbf{E}_{#1}\left[#2\right]}
\renewcommand{\emptyset}{\varnothing}
\newcommand{\ol}{\overline}
\newcommand{\argmin}{\mathop{\arg\min}}
\newcommand{\argmax}{\mathop{\arg\max}}
\renewcommand{\abs}[1]{\qty|#1|}
\newcommand{\defeq}{\triangleq} % triangle over =
\def\deq{\xlongequal{def}} % 'def' over =
\def\LHS{\text{LHS}}
\def\RHS{\text{RHS}}
\def\angbr#1{\langle#1\rangle} % <x>
\newcommand\set[1]{\qty{#1}}

\def\Esolve{\textcolor{blue}{Solve: }}
\def\Eproof{\textcolor{blue}{Proof: }}
\def\case#1{\textcolor{blue}{Case \uppercase\expandafter{\romannumeral#1}: }}
\def\card#1{\mbox{Card}(#1)}

% \newmdtheoremenv{lemma}{Lemma}
% \newmdtheoremenv{theorem}{\textcolor{red}{Theorem}}
% \newmdtheoremenv{defi}{\textcolor{blue}{Definition}}
\newtheorem{lemma}{Lemma}
\newtheorem{prp}{Proposition}
\newtheorem{thm}{Theorem}
\newtheorem{defi}{Defination}

\newenvironment{md}{\begin{mdframed}}{\end{mdframed}}
\usepackage{fancyhdr}
\pagestyle{fancy}
% \fancypagestyle{mainFancy}{
%     \fancyhf{}
%     \renewcommand\headrulewidth{.5pt}       % 页眉横线
%     \renewcommand\footrulewidth{0pt}
%     \fancyhead[OC]{\fzkai{\leftmark}}       % 页眉章标题
%     \fancyhead[EC]{\fzkai{\@title}}         % 页眉文章题目
% 	\lhead{\fzkai{author}}
%     \fancyhead[OR,EL]{\thepage}                 % 页眉编号
%     \fancyfoot[r]{\thumb} % 将拇指放到没有被使用的页眉或页脚处
% }
\fancyhf{}
\fancyhead[L]{\slshape{Haoyu Zhen}}
\fancyfoot[C]{\thepage}
\fancyhead[R]{\slshape{Available at \href{https://github.com/anyeZHY/AI2613-Homework}{https://github.com/anyeZHY/AI2613-Homework}}}

\graphicspath{{figures/}}

% \begin{document}
% \title{<++>}
\author{Haoyu Zhen}
% \maketitle
\setlength{\parindent}{0pt}
\setstretch{1.2}
% \end{document}


% \input{/Users/anye_zhenhaoyu/Desktop/ln_preamble.tex}

% %! Tex program = xelatex
% \documentclass{article}
%中文
%\usepackage[UTF8]{ctex}
%数学公式
\usepackage{amsmath,amssymb}
%\usepackage{ntheorem}
% \usepackage[framemethod=TikZ]{mdframed}
\usepackage{amsthm}
%边界
\usepackage[letterpaper,top=2.5cm,bottom=2.5cm,left=3cm,right=3cm,marginparwidth=1.75cm]{geometry}%table package
%Table
\usepackage{multirow,booktabs}
\usepackage{makecell}
%字体颜色
\usepackage{color}
\usepackage[dvipsnames]{xcolor}  % 更全的色系
%代码
\usepackage[OT1]{fontenc}
% MATLAB 代码风格
%\usepackage[framed,numbered,autolinebreaks,useliterate]{/Users/anye_zhenhaoyu/Desktop/Latex/mcode}
\usepackage{listings}
\usepackage{algorithm}
\usepackage{algorithmic}
\usepackage{pythonhighlight} % Python
%插图
\usepackage{graphicx}
%改变item格式
\usepackage{enumerate}
%物理
\usepackage{physics}
%extra arrows
\usepackage{extarrows}
% caption(居中指令)
%\usepackage[justification=centering]{caption}
\usepackage{caption}
% htpb
\usepackage{stfloats}
% pdf 拼接
\usepackage{pdfpages}
% 超链接url
\usepackage{url}
% \usepackage{tikz}
\usepackage{pgfplots}
\pgfplotsset{compat=newest}
\usepackage[colorlinks=true, allcolors=blue]{hyperref}
\usepackage{setspace}

% --------------definations-------------- %
\def\*#1{\boldsymbol{#1}}
\def\+#1{\mathcal{#1}} 
\def\-#1{\bar{#1}}
% Domains
\def\expect{\mathbb{E}}
\def\RR{\mathbb{R}}
\def\CC{\mathbb{C}}
\def\NN{\mathbb{N}}
\def\ZZ{\mathbb{Z}}
% Newcommand
\newcommand{\inner}[2]{\langle #1,#2\rangle} 
\newcommand{\numP}{\#\mathbf{P}} 
\renewcommand{\P}{\mathbf{P}}
\newcommand{\Var}[2][]{\mathbf{Var}_{#1}\left[#2\right]}
\newcommand{\E}[2][]{\mathbf{E}_{#1}\left[#2\right]}
\renewcommand{\emptyset}{\varnothing}
\newcommand{\ol}{\overline}
\newcommand{\argmin}{\mathop{\arg\min}}
\newcommand{\argmax}{\mathop{\arg\max}}
\renewcommand{\abs}[1]{\qty|#1|}
\newcommand{\defeq}{\triangleq} % triangle over =
\def\deq{\xlongequal{def}} % 'def' over =
\def\LHS{\text{LHS}}
\def\RHS{\text{RHS}}
\def\angbr#1{\langle#1\rangle} % <x>
\newcommand\set[1]{\qty{#1}}

\def\Esolve{\textcolor{blue}{Solve: }}
\def\Eproof{\textcolor{blue}{Proof: }}
\def\case#1{\textcolor{blue}{Case \uppercase\expandafter{\romannumeral#1}: }}
\def\card#1{\mbox{Card}(#1)}

% \newmdtheoremenv{lemma}{Lemma}
% \newmdtheoremenv{theorem}{\textcolor{red}{Theorem}}
% \newmdtheoremenv{defi}{\textcolor{blue}{Definition}}
\newtheorem{lemma}{Lemma}
\newtheorem{prp}{Proposition}
\newtheorem{thm}{Theorem}
\newtheorem{defi}{Defination}

\newenvironment{md}{\begin{mdframed}}{\end{mdframed}}
\usepackage{fancyhdr}
\pagestyle{fancy}
% \fancypagestyle{mainFancy}{
%     \fancyhf{}
%     \renewcommand\headrulewidth{.5pt}       % 页眉横线
%     \renewcommand\footrulewidth{0pt}
%     \fancyhead[OC]{\fzkai{\leftmark}}       % 页眉章标题
%     \fancyhead[EC]{\fzkai{\@title}}         % 页眉文章题目
% 	\lhead{\fzkai{author}}
%     \fancyhead[OR,EL]{\thepage}                 % 页眉编号
%     \fancyfoot[r]{\thumb} % 将拇指放到没有被使用的页眉或页脚处
% }
\fancyhf{}
\fancyhead[L]{\slshape{Haoyu Zhen}}
\fancyfoot[C]{\thepage}
\fancyhead[R]{\slshape{Available at \href{https://github.com/anyeZHY/AI2613-Homework}{https://github.com/anyeZHY/AI2613-Homework}}}

\graphicspath{{figures/}}

% \begin{document}
% \title{<++>}
\author{Haoyu Zhen}
% \maketitle
\setlength{\parindent}{0pt}
\setstretch{1.2}
% \end{document}


% \input{/path/ln_preamble.tex}

% On my MAC's Desktop
%! Tex program = xelatex
% \documentclass{article}
%中文
%\usepackage[UTF8]{ctex}
%数学公式
\usepackage{amsmath,amssymb}
%\usepackage{ntheorem}
% \usepackage[framemethod=TikZ]{mdframed}
\usepackage{amsthm}
%边界
\usepackage[letterpaper,top=2.5cm,bottom=2.5cm,left=3cm,right=3cm,marginparwidth=1.75cm]{geometry}%table package
%Table
\usepackage{multirow,booktabs}
\usepackage{makecell}
%字体颜色
\usepackage{color}
\usepackage[dvipsnames]{xcolor}  % 更全的色系
%代码
\usepackage[OT1]{fontenc}
% MATLAB 代码风格
%\usepackage[framed,numbered,autolinebreaks,useliterate]{/Users/anye_zhenhaoyu/Desktop/Latex/mcode}
\usepackage{listings}
\usepackage{algorithm}
\usepackage{algorithmic}
\usepackage{pythonhighlight} % Python
%插图
\usepackage{graphicx}
%改变item格式
\usepackage{enumerate}
%物理
\usepackage{physics}
%extra arrows
\usepackage{extarrows}
% caption(居中指令)
%\usepackage[justification=centering]{caption}
\usepackage{caption}
% htpb
\usepackage{stfloats}
% pdf 拼接
\usepackage{pdfpages}
% 超链接url
\usepackage{url}
% \usepackage{tikz}
\usepackage{pgfplots}
\pgfplotsset{compat=newest}
\usepackage[colorlinks=true, allcolors=blue]{hyperref}
\usepackage{setspace}

% --------------definations-------------- %
\def\*#1{\boldsymbol{#1}}
\def\+#1{\mathcal{#1}} 
\def\-#1{\bar{#1}}
% Domains
\def\expect{\mathbb{E}}
\def\RR{\mathbb{R}}
\def\CC{\mathbb{C}}
\def\NN{\mathbb{N}}
\def\ZZ{\mathbb{Z}}
% Newcommand
\newcommand{\inner}[2]{\langle #1,#2\rangle} 
\newcommand{\numP}{\#\mathbf{P}} 
\renewcommand{\P}{\mathbf{P}}
\newcommand{\Var}[2][]{\mathbf{Var}_{#1}\left[#2\right]}
\newcommand{\E}[2][]{\mathbf{E}_{#1}\left[#2\right]}
\renewcommand{\emptyset}{\varnothing}
\newcommand{\ol}{\overline}
\newcommand{\argmin}{\mathop{\arg\min}}
\newcommand{\argmax}{\mathop{\arg\max}}
\renewcommand{\abs}[1]{\qty|#1|}
\newcommand{\defeq}{\triangleq} % triangle over =
\def\deq{\xlongequal{def}} % 'def' over =
\def\LHS{\text{LHS}}
\def\RHS{\text{RHS}}
\def\angbr#1{\langle#1\rangle} % <x>
\newcommand\set[1]{\qty{#1}}

\def\Esolve{\textcolor{blue}{Solve: }}
\def\Eproof{\textcolor{blue}{Proof: }}
\def\case#1{\textcolor{blue}{Case \uppercase\expandafter{\romannumeral#1}: }}
\def\card#1{\mbox{Card}(#1)}

% \newmdtheoremenv{lemma}{Lemma}
% \newmdtheoremenv{theorem}{\textcolor{red}{Theorem}}
% \newmdtheoremenv{defi}{\textcolor{blue}{Definition}}
\newtheorem{lemma}{Lemma}
\newtheorem{prp}{Proposition}
\newtheorem{thm}{Theorem}
\newtheorem{defi}{Defination}

\newenvironment{md}{\begin{mdframed}}{\end{mdframed}}
\usepackage{fancyhdr}
\pagestyle{fancy}
% \fancypagestyle{mainFancy}{
%     \fancyhf{}
%     \renewcommand\headrulewidth{.5pt}       % 页眉横线
%     \renewcommand\footrulewidth{0pt}
%     \fancyhead[OC]{\fzkai{\leftmark}}       % 页眉章标题
%     \fancyhead[EC]{\fzkai{\@title}}         % 页眉文章题目
% 	\lhead{\fzkai{author}}
%     \fancyhead[OR,EL]{\thepage}                 % 页眉编号
%     \fancyfoot[r]{\thumb} % 将拇指放到没有被使用的页眉或页脚处
% }
\fancyhf{}
\fancyhead[L]{\slshape{Haoyu Zhen}}
\fancyfoot[C]{\thepage}
\fancyhead[R]{\slshape{Available at \href{https://github.com/anyeZHY/AI2613-Homework}{https://github.com/anyeZHY/AI2613-Homework}}}

\graphicspath{{figures/}}

% \begin{document}
% \title{<++>}
\author{Haoyu Zhen}
% \maketitle
\setlength{\parindent}{0pt}
\setstretch{1.2}
% \end{document}



\graphicspath{{figures/}}

\begin{document}
% \tableofcontents
\title{\vspace{-1cm}Homework 4}
\maketitle
\section{Doob's martingale inequality}
For any given $n$:

Consider the stopping time $\tau=\argmin_{0\le t\le n}\{X_t\ge \alpha\}$ or $\tau=n$ if  $X_t<\alpha$ for all  $0\le t\le n$. Ostensively, $\max_{0\le t\le n}X_t\ge \alpha\iff (\exists k) X_k\ge \alpha \iff X_\tau\ge \alpha$. Thus we have:
\[
	\Pr[\max_{0\le t\le n}X_t\ge \alpha]
	=
	\Pr[X_\tau\ge \alpha]
	\le 
	\frac{\mathbb{E}[X_\tau]}{\alpha}
\] where the inequality holds by Markov Ineq.

By defination, $\tau\le n$, which means $\mathbb{E}[X_\tau]=\mathbb{E_0}$ by Optional Stopping Theorem.
Then we obtain $\Pr[\max_{0\le t\le n}X_t\ge \alpha]\le \frac{\mathbb{E}[X_0]}{\alpha}$.

\section{Biased one-dimensional random walk}
\subsection*{Subproblem 1}
By defination,
\[
	\begin{aligned}
		\mathbb{E}\qty[S_{t+1}|\overline{Z_{i,n}}]
		&=
		\mathbb{E}\qty[S_t+Z_{t+1}+2p-1|\overline{Z_{i,n}}]
		\\&=
		S_t+2p-1+(1-p)-p
		\\&=
		S_t	
	\end{aligned}
.\] 
\subsection*{Subproblem 2}
Simmilarly,
\[
    \begin{aligned}
    	\mathbb{E}\qty[P_{t+1}|\overline{Z_{i,n}}]
		&=
		\mathbb{E}\qty[P_t\qty(\frac{p}{1-p})^{Z_{t+1}}|\overline{Z_{i,n}}]
		\\&=
		P_t\qty[p\times\frac{1-p}{p}+(1-p)\times\frac{p}{1-p}]
		\\&=
		P_t
    \end{aligned}
.\] 

\subsection*{Subproblem 3}
We define $p_a=\Pr[X_\tau=a]$ and $p_b=\Pr[X_\tau=b]=1-p_a$.

Now we want to show that $\{S_i\}$ and $\{P_i\}$ satisfy the conditions of OST theorem. Firstly we have 
\[
	\Pr(\text{ending within the next $a+b$ steps})\ge \qty[\max(p,1-p)]^{-a-b}
\] which means $\Pr[\tau<\infty]=1$.
Dividing the time into consecutive periods in this manner, we have $\mathbb{E}[\tau]<\infty$. Obviously, $\abs{P_i}$ is bounded. Also we have:  $\mathbb{E}[\abs{S_{t+1}-S_t}\+F_t]=p(2-2p)+(1-p)2p=4p(1-p)$.

By OST Thm., $\mathbb{E}[S_\tau]=\mathbb{E}[S_1]$ and  $\mathbb{E}[P_\tau]=\mathbb{E}[P_1]$. These propositions entail
\[
	(2p-1)\mathbb{E}[\tau]+bp_b-ap_a=0
	\qand
	\qty(\frac{p}{1-p})^{b}p_b+\qty(\frac{p}{1-p})^{-a}p_a=1
.\] 
Thus 
\[
    p_a=
	\frac
	{1-\qty[\flatfrac{p}{(1-p)}]^b}
	{\qty[\flatfrac{(1-p)}{p}]^{a}-\qty[\flatfrac{p}{(1-p)}]^b}
	\qand
	p_b=
	\frac
	{\qty[\flatfrac{(1-p)}{p}]^{a}-1}
	{\qty[\flatfrac{(1-p)}{p}]^{a}-\qty[\flatfrac{p}{(1-p)}]^b}
.\] 
Finally,
\[
	\mathbb{E}[\tau]=\frac{(a+b)(1-p)^bp^a-ap^{a+b}-b(1-p)^{a+b}}{(2p-1)\qty[(1-p)^{a+b}-p^{a+b}]}
.\] 
Remark: $\lim_{p\to\flatfrac{1}{2}}\mathbb{E}[\tau]=ab$.

\section{Longest common subsequence}
Talk with Yilin Sun and Liyuan Mao.
\subsection*{Subproblem 1}
When $n=2$,  
\[
	\mathbb{E}[X]
	=
	\frac{2\times(2+1+1+0)+2\times(1+2+1+1)}{16}
	=
	\frac{9}{8}\ge 2\times\frac{9}{16}
.\]
Similarly, if $n=3$, 
\[
	\mathbb{E}[X]=\frac{29}{16}\ge 3\times\frac{29}{48}
.\]
Since we could divide every strings into continuous subsequences whose length is $2$ or $3$, 
\[
	c_1=\frac{9}{16}
.\]
For any given $n$:
\[
	\Pr[X\ge k]
	\le 
	\frac{2^{2n-k}}{2^{2n}}{n \choose k}^2
	\approx
	{\frac{n}{2\pi k(n-k)}}\qty[\frac{n^n}{k^k(n-k)^{n-k}}]^2\frac{1}{2^k}
.\]
Let $k=cn$. We obtain
\[
    \Pr[X\ge cn]
	\le
	\frac{1}{2\pi nc(1-c)}\qty[\frac{1}{\sqrt{2}^cc^c(1-c)^{1-c}}]^{2n}
.\] 
Set $c=0.99$:
\[
    \Pr[X\ge 0.8n]
	\le
	\frac{51}{\pi n}(0.76)^n
	\text{ when } n\to\infty
.\] 
Thus $c_2$ exists.

\subsection*{Subproblem 2}

Denote function $f(x_1,\cdots,x_n,y_1,\cdots,y_n)$ as the LCS's length of the 2 sequences $\*x$ and $\*y$.
Obviously $f$ is 1-Lipschitz.
By McDiarmid's Inequality:
\[
	\Pr(\abs{X-\mathbb{E}[X]}\ge t)\le 2e^{-\flatfrac{2t^2}{n}}
.\] 

\end{document}

