% ! Tex program = xelatex
\documentclass{article}
% 中文
% \usepackage[UTF8]{ctex}

% For more choices
% %! Tex program = xelatex
% \documentclass{article}
%中文
%\usepackage[UTF8]{ctex}
%数学公式
\usepackage{amsmath,amssymb}
%\usepackage{ntheorem}
% \usepackage[framemethod=TikZ]{mdframed}
\usepackage{amsthm}
%边界
\usepackage[letterpaper,top=2.5cm,bottom=2.5cm,left=3cm,right=3cm,marginparwidth=1.75cm]{geometry}%table package
%Table
\usepackage{multirow,booktabs}
\usepackage{makecell}
%字体颜色
\usepackage{color}
\usepackage[dvipsnames]{xcolor}  % 更全的色系
%代码
\usepackage[OT1]{fontenc}
% MATLAB 代码风格
%\usepackage[framed,numbered,autolinebreaks,useliterate]{/Users/anye_zhenhaoyu/Desktop/Latex/mcode}
\usepackage{listings}
\usepackage{algorithm}
\usepackage{algorithmic}
\usepackage{pythonhighlight} % Python
%插图
\usepackage{graphicx}
%改变item格式
\usepackage{enumerate}
%物理
\usepackage{physics}
%extra arrows
\usepackage{extarrows}
% caption(居中指令)
%\usepackage[justification=centering]{caption}
\usepackage{caption}
% htpb
\usepackage{stfloats}
% pdf 拼接
\usepackage{pdfpages}
% 超链接url
\usepackage{url}
% \usepackage{tikz}
\usepackage{pgfplots}
\pgfplotsset{compat=newest}
\usepackage[colorlinks=true, allcolors=blue]{hyperref}
\usepackage{setspace}

% --------------definations-------------- %
\def\*#1{\boldsymbol{#1}}
\def\+#1{\mathcal{#1}} 
\def\-#1{\mathrm{#1}}
\def\=#1{\mathbb{#1}}
% Domains
\def\RR{\mathbb{R}}
\def\CC{\mathbb{C}}
\def\NN{\mathbb{N}}
\def\ZZ{\mathbb{Z}}
% Newcommand
\newcommand{\inner}[2]{\langle #1,#2\rangle} 
\newcommand{\numP}{\#\mathbf{P}} 
\renewcommand{\P}{\mathbf{P}}
\newcommand{\Var}[2][]{\mathbf{Var}_{#1}\left[#2\right]}
\newcommand{\E}[2][]{\mathbf{E}_{#1}\left[#2\right]}
\renewcommand{\emptyset}{\varnothing}
\newcommand{\ol}{\overline}
\newcommand{\argmin}{\mathop{\arg\min}}
\newcommand{\argmax}{\mathop{\arg\max}}
\renewcommand{\abs}[1]{\qty|#1|}
\newcommand{\defeq}{\triangleq} % triangle over =
\def\deq{\xlongequal{def}} % 'def' over =
\def\LHS{\text{LHS}}
\def\RHS{\text{RHS}}
\def\angbr#1{\langle#1\rangle} % <x>
\newcommand\set[1]{\qty{#1}}

\def\Esolve{\textcolor{blue}{Solve: }}
\def\Eproof{\textcolor{blue}{Proof: }}
\def\case#1{\textcolor{blue}{Case \uppercase\expandafter{\romannumeral#1}: }}
\def\card#1{\mbox{Card}(#1)}

% \newmdtheoremenv{lemma}{Lemma}
% \newmdtheoremenv{theorem}{\textcolor{red}{Theorem}}
% \newmdtheoremenv{defi}{\textcolor{blue}{Definition}}
\newtheorem{lemma}{Lemma}
\newtheorem{prp}{Proposition}
\newtheorem{thm}{Theorem}
\newtheorem{defi}{Defination}

\newenvironment{md}{\begin{mdframed}}{\end{mdframed}}
\usepackage{fancyhdr}
\pagestyle{fancy}
% \fancypagestyle{mainFancy}{
%     \fancyhf{}
%     \renewcommand\headrulewidth{.5pt}       % 页眉横线
%     \renewcommand\footrulewidth{0pt}
%     \fancyhead[OC]{\fzkai{\leftmark}}       % 页眉章标题
%     \fancyhead[EC]{\fzkai{\@title}}         % 页眉文章题目
% 	\lhead{\fzkai{author}}
%     \fancyhead[OR,EL]{\thepage}                 % 页眉编号
%     \fancyfoot[r]{\thumb} % 将拇指放到没有被使用的页眉或页脚处
% }
\fancyhf{}
\fancyhead[L]{\slshape{Haoyu Zhen}}
\fancyfoot[C]{\thepage}
\fancyhead[R]{\slshape{Available at \href{https://github.com/anyeZHY/AI2613-Homework}{https://github.com/anyeZHY/AI2613-Homework}}}

\graphicspath{{figures/}}

% \begin{document}
% \title{<++>}
\author{Haoyu Zhen}
% \maketitle
\setlength{\parindent}{0pt}
\setstretch{1.3}
% \end{document}


% \input{/Users/anye_zhenhaoyu/Desktop/ln_preamble.tex}

% %! Tex program = xelatex
% \documentclass{article}
%中文
%\usepackage[UTF8]{ctex}
%数学公式
\usepackage{amsmath,amssymb}
%\usepackage{ntheorem}
% \usepackage[framemethod=TikZ]{mdframed}
\usepackage{amsthm}
%边界
\usepackage[letterpaper,top=2.5cm,bottom=2.5cm,left=3cm,right=3cm,marginparwidth=1.75cm]{geometry}%table package
%Table
\usepackage{multirow,booktabs}
\usepackage{makecell}
%字体颜色
\usepackage{color}
\usepackage[dvipsnames]{xcolor}  % 更全的色系
%代码
\usepackage[OT1]{fontenc}
% MATLAB 代码风格
%\usepackage[framed,numbered,autolinebreaks,useliterate]{/Users/anye_zhenhaoyu/Desktop/Latex/mcode}
\usepackage{listings}
\usepackage{algorithm}
\usepackage{algorithmic}
\usepackage{pythonhighlight} % Python
%插图
\usepackage{graphicx}
%改变item格式
\usepackage{enumerate}
%物理
\usepackage{physics}
%extra arrows
\usepackage{extarrows}
% caption(居中指令)
%\usepackage[justification=centering]{caption}
\usepackage{caption}
% htpb
\usepackage{stfloats}
% pdf 拼接
\usepackage{pdfpages}
% 超链接url
\usepackage{url}
% \usepackage{tikz}
\usepackage{pgfplots}
\pgfplotsset{compat=newest}
\usepackage[colorlinks=true, allcolors=blue]{hyperref}
\usepackage{setspace}

% --------------definations-------------- %
\def\*#1{\boldsymbol{#1}}
\def\+#1{\mathcal{#1}} 
\def\-#1{\mathrm{#1}}
\def\=#1{\mathbb{#1}}
% Domains
\def\RR{\mathbb{R}}
\def\CC{\mathbb{C}}
\def\NN{\mathbb{N}}
\def\ZZ{\mathbb{Z}}
% Newcommand
\newcommand{\inner}[2]{\langle #1,#2\rangle} 
\newcommand{\numP}{\#\mathbf{P}} 
\renewcommand{\P}{\mathbf{P}}
\newcommand{\Var}[2][]{\mathbf{Var}_{#1}\left[#2\right]}
\newcommand{\E}[2][]{\mathbf{E}_{#1}\left[#2\right]}
\renewcommand{\emptyset}{\varnothing}
\newcommand{\ol}{\overline}
\newcommand{\argmin}{\mathop{\arg\min}}
\newcommand{\argmax}{\mathop{\arg\max}}
\renewcommand{\abs}[1]{\qty|#1|}
\newcommand{\defeq}{\triangleq} % triangle over =
\def\deq{\xlongequal{def}} % 'def' over =
\def\LHS{\text{LHS}}
\def\RHS{\text{RHS}}
\def\angbr#1{\langle#1\rangle} % <x>
\newcommand\set[1]{\qty{#1}}

\def\Esolve{\textcolor{blue}{Solve: }}
\def\Eproof{\textcolor{blue}{Proof: }}
\def\case#1{\textcolor{blue}{Case \uppercase\expandafter{\romannumeral#1}: }}
\def\card#1{\mbox{Card}(#1)}

% \newmdtheoremenv{lemma}{Lemma}
% \newmdtheoremenv{theorem}{\textcolor{red}{Theorem}}
% \newmdtheoremenv{defi}{\textcolor{blue}{Definition}}
\newtheorem{lemma}{Lemma}
\newtheorem{prp}{Proposition}
\newtheorem{thm}{Theorem}
\newtheorem{defi}{Defination}

\newenvironment{md}{\begin{mdframed}}{\end{mdframed}}
\usepackage{fancyhdr}
\pagestyle{fancy}
% \fancypagestyle{mainFancy}{
%     \fancyhf{}
%     \renewcommand\headrulewidth{.5pt}       % 页眉横线
%     \renewcommand\footrulewidth{0pt}
%     \fancyhead[OC]{\fzkai{\leftmark}}       % 页眉章标题
%     \fancyhead[EC]{\fzkai{\@title}}         % 页眉文章题目
% 	\lhead{\fzkai{author}}
%     \fancyhead[OR,EL]{\thepage}                 % 页眉编号
%     \fancyfoot[r]{\thumb} % 将拇指放到没有被使用的页眉或页脚处
% }
\fancyhf{}
\fancyhead[L]{\slshape{Haoyu Zhen}}
\fancyfoot[C]{\thepage}
\fancyhead[R]{\slshape{Available at \href{https://github.com/anyeZHY/AI2613-Homework}{https://github.com/anyeZHY/AI2613-Homework}}}

\graphicspath{{figures/}}

% \begin{document}
% \title{<++>}
\author{Haoyu Zhen}
% \maketitle
\setlength{\parindent}{0pt}
\setstretch{1.3}
% \end{document}


% \input{/path/ln_preamble.tex}

% On my MAC's Desktop
%! Tex program = xelatex
% \documentclass{article}
%中文
%\usepackage[UTF8]{ctex}
%数学公式
\usepackage{amsmath,amssymb}
%\usepackage{ntheorem}
% \usepackage[framemethod=TikZ]{mdframed}
\usepackage{amsthm}
%边界
\usepackage[letterpaper,top=2.5cm,bottom=2.5cm,left=3cm,right=3cm,marginparwidth=1.75cm]{geometry}%table package
%Table
\usepackage{multirow,booktabs}
\usepackage{makecell}
%字体颜色
\usepackage{color}
\usepackage[dvipsnames]{xcolor}  % 更全的色系
%代码
\usepackage[OT1]{fontenc}
% MATLAB 代码风格
%\usepackage[framed,numbered,autolinebreaks,useliterate]{/Users/anye_zhenhaoyu/Desktop/Latex/mcode}
\usepackage{listings}
\usepackage{algorithm}
\usepackage{algorithmic}
\usepackage{pythonhighlight} % Python
%插图
\usepackage{graphicx}
%改变item格式
\usepackage{enumerate}
%物理
\usepackage{physics}
%extra arrows
\usepackage{extarrows}
% caption(居中指令)
%\usepackage[justification=centering]{caption}
\usepackage{caption}
% htpb
\usepackage{stfloats}
% pdf 拼接
\usepackage{pdfpages}
% 超链接url
\usepackage{url}
% \usepackage{tikz}
\usepackage{pgfplots}
\pgfplotsset{compat=newest}
\usepackage[colorlinks=true, allcolors=blue]{hyperref}
\usepackage{setspace}

% --------------definations-------------- %
\def\*#1{\boldsymbol{#1}}
\def\+#1{\mathcal{#1}} 
\def\-#1{\mathrm{#1}}
\def\=#1{\mathbb{#1}}
% Domains
\def\RR{\mathbb{R}}
\def\CC{\mathbb{C}}
\def\NN{\mathbb{N}}
\def\ZZ{\mathbb{Z}}
% Newcommand
\newcommand{\inner}[2]{\langle #1,#2\rangle} 
\newcommand{\numP}{\#\mathbf{P}} 
\renewcommand{\P}{\mathbf{P}}
\newcommand{\Var}[2][]{\mathbf{Var}_{#1}\left[#2\right]}
\newcommand{\E}[2][]{\mathbf{E}_{#1}\left[#2\right]}
\renewcommand{\emptyset}{\varnothing}
\newcommand{\ol}{\overline}
\newcommand{\argmin}{\mathop{\arg\min}}
\newcommand{\argmax}{\mathop{\arg\max}}
\renewcommand{\abs}[1]{\qty|#1|}
\newcommand{\defeq}{\triangleq} % triangle over =
\def\deq{\xlongequal{def}} % 'def' over =
\def\LHS{\text{LHS}}
\def\RHS{\text{RHS}}
\def\angbr#1{\langle#1\rangle} % <x>
\newcommand\set[1]{\qty{#1}}

\def\Esolve{\textcolor{blue}{Solve: }}
\def\Eproof{\textcolor{blue}{Proof: }}
\def\case#1{\textcolor{blue}{Case \uppercase\expandafter{\romannumeral#1}: }}
\def\card#1{\mbox{Card}(#1)}

% \newmdtheoremenv{lemma}{Lemma}
% \newmdtheoremenv{theorem}{\textcolor{red}{Theorem}}
% \newmdtheoremenv{defi}{\textcolor{blue}{Definition}}
\newtheorem{lemma}{Lemma}
\newtheorem{prp}{Proposition}
\newtheorem{thm}{Theorem}
\newtheorem{defi}{Defination}

\newenvironment{md}{\begin{mdframed}}{\end{mdframed}}
\usepackage{fancyhdr}
\pagestyle{fancy}
% \fancypagestyle{mainFancy}{
%     \fancyhf{}
%     \renewcommand\headrulewidth{.5pt}       % 页眉横线
%     \renewcommand\footrulewidth{0pt}
%     \fancyhead[OC]{\fzkai{\leftmark}}       % 页眉章标题
%     \fancyhead[EC]{\fzkai{\@title}}         % 页眉文章题目
% 	\lhead{\fzkai{author}}
%     \fancyhead[OR,EL]{\thepage}                 % 页眉编号
%     \fancyfoot[r]{\thumb} % 将拇指放到没有被使用的页眉或页脚处
% }
\fancyhf{}
\fancyhead[L]{\slshape{Haoyu Zhen}}
\fancyfoot[C]{\thepage}
\fancyhead[R]{\slshape{Available at \href{https://github.com/anyeZHY/AI2613-Homework}{https://github.com/anyeZHY/AI2613-Homework}}}

\graphicspath{{figures/}}

% \begin{document}
% \title{<++>}
\author{Haoyu Zhen}
% \maketitle
\setlength{\parindent}{0pt}
\setstretch{1.3}
% \end{document}



\graphicspath{{figures/}}

\begin{document}
% \tableofcontents
\title{Homework 2}
\maketitle
\section{Optimal Coupling}
Assume $\Omega=\qty{1,2, \cdots,n}$.
Our goal is to fill the $n\times n$ probabilty table, which means we need to determine every value of $P_{i,j}$ such that
\begin{equation}
	\sum_i P(i,j)=\nu(j)\mbox{ and }\sum_j P_{i,j}=\mu(i)
	\label{constrain}
\end{equation}

Constructe $\*P$ as follows:
\begin{enumerate}
	\item
	      Let $\Pr_{(X,Y)\sim \omega}(X=Y=i)=\min[\mu(i),\nu(i)]$.
	\item
	      Then
	      \[
		      \mbox{If }i\ne j,\ P_{i,j}=\frac{(\mu(i)-P_{i,i})(\nu(j)-P_{j,j})}{1-\sum_k{P_{k,k}}}
		      .\]
\end{enumerate}
Reference: \href{https://courses.cs.duke.edu/spring13/compsci590.2/slides/lec5.pdf}{Markov Chains and Coupling, Duke University}. And the proof is not mentioned in this lecture notes.

Now we want to prove that the construction satisfies the EQ. \ref{constrain}.
If $\nu(i)=P_{i,i}$:
\[
	\begin{aligned}
		 & P_{i,i}+
		\sum_{j\ne i}\frac{(\mu(i)-P_{i,i})(\nu(j)-P_{j,j})}{1-\sum_k{P_{k,k}}}
		\\[5pt]=&
		P_{i,i}+
		(\mu(i)-P_{i,i})\frac{\sum_{j\ne i}(\nu(j)-P_{j,j})}{1-\sum_k{P_{k,k}}}
		\\[5pt]=&
		P_{i,i}+
		(\mu(i)-P_{i,i})\frac{\sum_{j}(\nu(j)-P_{j,j})}{1-\sum_k{P_{k,k}}}
		\\[5pt]=&
		P_{i,i}+
		(\mu(i)-P_{i,i})\frac{1-\sum_{j}P_{j,j}}{1-\sum_k{P_{k,k}}}
		\\[5pt]=&
		\mu(i)
	\end{aligned}
	.\]
If $\mu(i)=P_{i,i}$, obviously  $\sum_jP_{i,j}=P_{i,i}=\mu(i)$. Similarly,  $\sum_iP_{i,j}=\nu(j)$. And the EQ.\ref{constrain} holds.

Thus
\[
	\Pr_{(X,Y)\sim\omega}[X\ne Y]=1-\sum_i\min(\mu(i),\nu(i))=\sum_i\qty[\mu(i)-\min(\mu(i),\nu(i))]=\max_{A\subset\Omega}\qty|\mu(A)-\nu(A)|=D_{TV}(\mu,\nu)
	.\]

\newpage
\section{Stochastic Dominance}
\subsection*{Problem 1}
``$\Longrightarrow$'': $\Pr[X=n]\ge \Pr[Y=n]$, which means that $p^n\ge q^n$. Thus $p\ge q$.
\\\\
``$\Longleftarrow$'', if $p\ge q$:

For simplicity, $\Pr[X\ge k]-\Pr[Y\ge k]\defeq A_k$. $A_0=0$ and $A_n=p^n-q^n$.
Then
\[
	A_{k+1}-A_{k}=q^k(1-q)^{n-k}-p^k(1-p)^{n-k}
	.\]
Let $\displaystyle k_0=\frac{n\log(\flatfrac{1-q}{1-p})}{\log[\flatfrac{p(1-q)}{q(1-p)}]}$.
\\[9pt]
If $k\le k_0$, then $A_{k+1}-A_k\ge 0$. And if $k_0\le k\le n$,
then $A_{k+1}-A_k\le 0.$
\\[6pt]
Thus $0=A_0\nearrow A_{k_0}\searrow A_{n}=p^n-q^n>0$ for some $k_0$, which means that $A_k>0$.
\\[6pt]
Finally \[\Pr[X\ge k]-\Pr[Y\ge k]\ge 0.\]

\subsection*{Problem 2}
Assume $\Omega=\qty{1,2, \cdots,n}$.
\\\\
``$\Longleftarrow$'': $P_{i,j}$ denote  $\omega(i,j)$.  $P_{i,j}=0$ if  $j>i$.
\[
	\begin{aligned}
		\Pr[X\ge k]=\sum_{i=k}^n\sum_{j=0}^iP_{i,j}
		\le
		\sum_{i=k}^n\sum_{j=k}^iP_{i,j}
		=
		\sum_{j=k}^i\sum_{i=k}^nP_{i,j}
		=
		\sum_{j=k}^n\sum_{i=0}^jP_{i,j}
		=
		\Pr[Y\ge k]
	\end{aligned}
	.\]

``$\Longrightarrow$'':

Design the coupling by the following steps:
\begin{enumerate}
	\item
	      $\forall i,j,\,\,P_{i,j}=0$.
	\item
	      Let  $(i,j)$ traverse: $(1,1),(2,1),(2,2),(3,1),(3,2),(3,3)\cdots,(n,n)$ and update  $P_{i,j}$:
	      \[
		      P_{i,j}=\min\qty{\qty[\mu(i)-\sum_{k=1}^iP_{i,k}],\,\qty[\nu(j)-\sum_{k=i}^jP_{k,j}]}
		      .\]
\end{enumerate}
For any $i$:

If $\exists j$, $P_{i,j}=\mu(i)-\sum_{k=1}^iP_{i,k}$, then $\sum_jP_{i,j}=\mu(i)$.

If not, then $\forall j,\, P_{i,j}=\nu(j)-\sum_{k=i}^jP_{k,j}$, which means $\sum_{k=i+1}^n\mu(k)=\sum_{k=i+1}^n\nu(k)$. Thus
\[
	\begin{aligned}
		\sum_{j=1}^nP_{i,j}
		&=
		\sum_{j=1}^iP_{i,j}
		=
		\sum_{j=1}^i\qty[\nu(j)-\sum_{k=i}^jP_{k,j}]
		\\[5pt]&=
		\sum_{k=1}^i\nu(j)-\sum_{k=1}^{i-1}\mu(k)
		=
		\sum_{k=1}^i\mu(j)-\sum_{k=1}^{i-1}\mu(k)
		\\[5pt]&=
		\mu(i)
	\end{aligned}
.\]

\subsection*{Problem 3}
I refer to the \href{http://chihaozhang.com/teaching/SP2021spring/notes/lec5-SP2021Spring.pdf}{lecture notes} last year.

\begin{table}[H]
	\centering
	\begin{tabular}{c|cc|c}
		& Not Connected & Connected & $G$
		\\ \hline
		Not Connected & $P_{11}$ & $P_{12}$ & $\Pr[\mbox{G is not connected}]$
		\\
		Connected & $P_{21}$ & $P_{22}$ & $\Pr[\mbox{G is connected}]$
		\\ \hline
		$H$ & $\Pr[\mbox{H is not connected}]$ & $\Pr[\mbox{H is connected}]$ &
	\end{tabular}
\end{table}

We generate $G\sim\+G(n,p)$ and $H\sim\+G(n,q)$ simultaneously. Let $r\sim\mathrm{U}(0,1)$. For every edge $(u,v)$:
\[
    \begin{cases}
		(u,v) \mbox{exisits in $G$ and $H$} &\mbox{if }r\in[0,q]
		\\
		(u,v) \mbox{exisits only in $G$} &\mbox{if }r\in(q,p]
		\\
		(u,v) \mbox{does not exisit} &\mbox{if }r\in(p,1]
    \end{cases}
.\]
Then for all $H$,  $H$ is the subgraph of $G$, which means that $P_{12}=0$. By 2 we have 
\[
	\Pr_{G\sim\+G(n,p)}\qty[\mbox{G is connected}]\ge\Pr_{H\sim\+G(n,q)}\qty[\mbox{H is connected}].
\]

\section{Total Variation Distance is Non-Increasing}

By coupling lemma, for every $X^t\sim\mu^t$ and  $Y^t\sim\pi$, we have a coupling such that  $\Delta(t)=\Pr(X^t\ne Y^t)$. Then we construct $X^{t+1}$ and  $Y^{t+1}$ by following rules:
 \[
	 \begin{cases}
		 X^{t+1}=X^t\mbox{ and }Y^{t+1}=Y^t & \mbox{, if }X^t=Y^t
		 \\
		 X^t\to X^{t+1}\mbox{ and }Y^t\to Y^{t+1} \mbox{ \textbf{independently}} & \mbox{, otherwise}
	 \end{cases}	 
.\] 
Thus
\[
	\Delta(t+1)\le \Pr(X^{t+1}\ne y^{t+1})\le\Pr(X^{t}\ne Y^{t})=\Delta(t)
.\] 
Reference: \href{https://courses.cs.duke.edu/spring13/compsci590.2/slides/lec5.pdf}{Markov Chains and Coupling, Duke University}.


\end{document}
