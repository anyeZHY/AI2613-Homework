\documentclass{article}
%! Tex program = xelatex
% \documentclass{article}
%中文
%\usepackage[UTF8]{ctex}
%数学公式
\usepackage{amsmath,amssymb}
%\usepackage{ntheorem}
% \usepackage[framemethod=TikZ]{mdframed}
\usepackage{amsthm}
%边界
\usepackage[letterpaper,top=2.5cm,bottom=2.5cm,left=3cm,right=3cm,marginparwidth=1.75cm]{geometry}%table package
%Table
\usepackage{multirow,booktabs}
\usepackage{makecell}
%字体颜色
\usepackage{color}
\usepackage[dvipsnames]{xcolor}  % 更全的色系
%代码
\usepackage[OT1]{fontenc}
% MATLAB 代码风格
%\usepackage[framed,numbered,autolinebreaks,useliterate]{/Users/anye_zhenhaoyu/Desktop/Latex/mcode}
\usepackage{listings}
\usepackage{algorithm}
\usepackage{algorithmic}
\usepackage{pythonhighlight} % Python
%插图
\usepackage{graphicx}
%改变item格式
\usepackage{enumerate}
%物理
\usepackage{physics}
%extra arrows
\usepackage{extarrows}
% caption(居中指令)
%\usepackage[justification=centering]{caption}
\usepackage{caption}
% htpb
\usepackage{stfloats}
% pdf 拼接
\usepackage{pdfpages}
% 超链接url
\usepackage{url}
% \usepackage{tikz}
\usepackage{pgfplots}
\pgfplotsset{compat=newest}
\usepackage[colorlinks=true, allcolors=blue]{hyperref}
\usepackage{setspace}

% --------------definations-------------- %
\def\*#1{\boldsymbol{#1}}
\def\+#1{\mathcal{#1}} 
\def\-#1{\bar{#1}}
% Domains
\def\expect{\mathbb{E}}
\def\RR{\mathbb{R}}
\def\CC{\mathbb{C}}
\def\NN{\mathbb{N}}
\def\ZZ{\mathbb{Z}}
% Newcommand
\newcommand{\inner}[2]{\langle #1,#2\rangle} 
\newcommand{\numP}{\#\mathbf{P}} 
\renewcommand{\P}{\mathbf{P}}
\newcommand{\Var}[2][]{\mathbf{Var}_{#1}\left[#2\right]}
\newcommand{\E}[2][]{\mathbf{E}_{#1}\left[#2\right]}
\renewcommand{\emptyset}{\varnothing}
\newcommand{\ol}{\overline}
\newcommand{\argmin}{\mathop{\arg\min}}
\newcommand{\argmax}{\mathop{\arg\max}}
\renewcommand{\abs}[1]{\qty|#1|}
\newcommand{\defeq}{\triangleq} % triangle over =
\def\deq{\xlongequal{def}} % 'def' over =
\def\LHS{\text{LHS}}
\def\RHS{\text{RHS}}
\def\angbr#1{\langle#1\rangle} % <x>
\newcommand\set[1]{\qty{#1}}

\def\Esolve{\textcolor{blue}{Solve: }}
\def\Eproof{\textcolor{blue}{Proof: }}
\def\case#1{\textcolor{blue}{Case \uppercase\expandafter{\romannumeral#1}: }}
\def\card#1{\mbox{Card}(#1)}

% \newmdtheoremenv{lemma}{Lemma}
% \newmdtheoremenv{theorem}{\textcolor{red}{Theorem}}
% \newmdtheoremenv{defi}{\textcolor{blue}{Definition}}
\newtheorem{lemma}{Lemma}
\newtheorem{prp}{Proposition}
\newtheorem{thm}{Theorem}
\newtheorem{defi}{Defination}

\newenvironment{md}{\begin{mdframed}}{\end{mdframed}}
\usepackage{fancyhdr}
\pagestyle{fancy}
% \fancypagestyle{mainFancy}{
%     \fancyhf{}
%     \renewcommand\headrulewidth{.5pt}       % 页眉横线
%     \renewcommand\footrulewidth{0pt}
%     \fancyhead[OC]{\fzkai{\leftmark}}       % 页眉章标题
%     \fancyhead[EC]{\fzkai{\@title}}         % 页眉文章题目
% 	\lhead{\fzkai{author}}
%     \fancyhead[OR,EL]{\thepage}                 % 页眉编号
%     \fancyfoot[r]{\thumb} % 将拇指放到没有被使用的页眉或页脚处
% }
\fancyhf{}
\fancyhead[L]{\slshape{Haoyu Zhen}}
\fancyfoot[C]{\thepage}
\fancyhead[R]{\slshape{Available at \href{https://github.com/anyeZHY/AI2613-Homework}{https://github.com/anyeZHY/AI2613-Homework}}}

\graphicspath{{figures/}}

% \begin{document}
% \title{<++>}
\author{Haoyu Zhen}
% \maketitle
\setlength{\parindent}{0pt}
\setstretch{1.2}
% \end{document}


\begin{document}
\title{Homework 1}
\maketitle

\section{Probability Space of Tossing Coins}
\subsection*{\centering Problem 1}
Obviously, $\varnothing\notin\{C_s\}_{\{0,1\}^n}$.
\\
By the fact that $C_s\subset\Omega$ and  $\mbox{Card}\qty(\{0,1\}^n)<+\infty$, we have $\bigcup_{s\in\{0,1\}^n}C_s\subset\Omega$.
\\
Also, $\forall\omega\in\Omega$, $\omega\in C_{(\omega_1, \dots,\omega_n)}$, which means $\Omega\subset\bigcup_{s\in\{0,1\}^n}C_s$.
\\
$\forall s_1,s_2\in\set{0,1}^n$, $\forall\omega_1\in C_{s_1}$ and $\omega_2\in C_{s_2}$, $\omega_1\ne \omega_2$. Then $C_{s_1}\cap C_{s_2}\ne \emptyset$.
\\
Finally, $\set{C_s}_{s\in\set{0,1}^n}$ forms a partition of  $\Omega$.

\subsection*{\centering Problem 2}
Firstly, I will prove a lemma and a proposition.
\begin{lemma}
	\label{fc}
	If $\+C$ forms a partition of $\Omega$, then  $2^\+C$ is a  $\sigma$-algebra.
\end{lemma}
\begin{proof}
		$\emptyset\in 2^\+C$.\\
		If $A\in 2^\+C$, then $\Omega-A\subset\+C\Longrightarrow\Omega-A\in 2^\+C$.\\
		If $A_i\in 2^\+C$, then $\bigcup_iA_i\subset\+C$, which means that $\bigcup_iA_i\in 2^\+C$.
\end{proof}
\begin{prp}
	$\+F_n=2^{\+C_n}$ where $\+C_n={\set{C_s}_{s\in\set{0,1}^n}}$. 
\end{prp}
\begin{proof}
	Trivially, $\+C_n\subset 2^{\+C_n}$. Now we suppose that $\exists$ $\sigma$-algebra  $\+F$ s.t. $\+C_n\subset\+F$:
\\
For all $A\in 2^{\+C_n}$, $A\subset\+C$. Then $A=\bigcup C_s$ for some $s$ or $A=\emptyset$. By the defination of $\sigma$-algebra, $A\in\+F$, which means  $2^{\+C_n}\in\+F$.
\\
Thus $2^{\+C_n}$ is the minimal  $\sigma$-algebra containing sets in $\+C$.
\end{proof}
Now we have $\text{Card}(\+F_n)=\text{Card}(2^{\+C_n})=\text{Card}(2^{\set{0,1}^n})=2^{2^n}$. We could say that $\+F_n\sim 2^{\set{0,1}^n}$ or $\+F_n$ and $2^{\set{0,1}^n}$ are \href{https://mathworld.wolfram.com/Equinumerous.html}{equinumerous}. So there exists a bijection between $\+F$ and  $2^{\set{0,1}^n}$.

\subsection*{\centering Problem 3}
By lemma \ref{fc}, $\forall n\in\NN$, $\+F_n=2^{\+C_n}$.

And obviously $\+F_n\ne \+F_{n+1}$ because $\+C_n\ne \+C_{n+1}$ and $\text{Card}(\+C_n)\ne \text{Card}(\+C_{n+1})$.

$\forall A\in\+F_n$, $A\subset\+C_n$. Assume that $A=\set{\omega\in\Omega\mid\omega_1=s_1, \cdots, \omega_n=s_n}$. Then
\[
	\set{\omega\in\Omega\mid\omega_1=s_1, \cdots, \omega_n=s_n,\omega_{n+1}=0}
	\cup
	\set{\omega\in\Omega\mid\omega_1=s_1, \cdots, \omega_n=s_n,\omega_{n+1}=1}
	=A
,\]
where the 2 elements of LHS are in $\+C_{n+1}$. Thus  $A\in\+C_{n+1}$.

Ultimately $\forall n\in\NN$, $\+F_n=2^{\+C_n}\subsetneq 2^{\+C_{n+1}}=\+F_{n+1}$.

\subsection*{\centering Problem 4}
By defination, $\+F_\infty=\bigcup_{n\ge 1}\+F_n$.

For all $A,B\in\+F_\infty$, $\exists m,n\in\NN$, $A\in\+F_m$ and $B\in\+F_n$. Let  $l\deq\max(m,n)$. Then  $A,B\in\+F_l$.

Thus $A^c,B^c\in\+F_l$ and  $A+B\in\+F_l\Longrightarrow A^c,B^c,A+B\in\+F_l$.

The proposition that $F_\infty$ is an algebra holds.

\begin{lemma}
	$\forall\omega\in\Omega$, $\forall n\in\NN$, $\omega\notin\+F_n$.
	\label{wf}
\end{lemma}
\begin{proof}
	$\forall A\in\+F_n$, $\card{A}=+\infty\ne \card{\set{\omega}}=1$
\end{proof}
Thus $2^{\Omega}\ne\+F_\infty$.


\subsection*{\centering Problem 5}
By lemma \ref{wf}, we have $\set{\omega}\ne\+F_\infty$. If not, there exists $n\in\NN$ such that  $\omega\notin\+F_n$, which lead to a contradiction.
Now we just need to prove that  $\set{\omega}\in\+B(\Omega)$.



\subsection*{}
\subsection*{}

\end{document}

