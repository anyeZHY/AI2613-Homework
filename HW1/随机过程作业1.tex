\documentclass{article}
%! Tex program = xelatex
% \documentclass{article}
%中文
%\usepackage[UTF8]{ctex}
%数学公式
\usepackage{amsmath,amssymb}
%\usepackage{ntheorem}
% \usepackage[framemethod=TikZ]{mdframed}
\usepackage{amsthm}
%边界
\usepackage[letterpaper,top=2.5cm,bottom=2.5cm,left=3cm,right=3cm,marginparwidth=1.75cm]{geometry}%table package
%Table
\usepackage{multirow,booktabs}
\usepackage{makecell}
%字体颜色
\usepackage{color}
\usepackage[dvipsnames]{xcolor}  % 更全的色系
%代码
\usepackage[OT1]{fontenc}
% MATLAB 代码风格
%\usepackage[framed,numbered,autolinebreaks,useliterate]{/Users/anye_zhenhaoyu/Desktop/Latex/mcode}
\usepackage{listings}
\usepackage{algorithm}
\usepackage{algorithmic}
\usepackage{pythonhighlight} % Python
%插图
\usepackage{graphicx}
%改变item格式
\usepackage{enumerate}
%物理
\usepackage{physics}
%extra arrows
\usepackage{extarrows}
% caption(居中指令)
%\usepackage[justification=centering]{caption}
\usepackage{caption}
% htpb
\usepackage{stfloats}
% pdf 拼接
\usepackage{pdfpages}
% 超链接url
\usepackage{url}
% \usepackage{tikz}
\usepackage{pgfplots}
\pgfplotsset{compat=newest}
\usepackage[colorlinks=true, allcolors=blue]{hyperref}
\usepackage{setspace}

% --------------definations-------------- %
\def\*#1{\boldsymbol{#1}}
\def\+#1{\mathcal{#1}} 
\def\-#1{\bar{#1}}
% Domains
\def\expect{\mathbb{E}}
\def\RR{\mathbb{R}}
\def\CC{\mathbb{C}}
\def\NN{\mathbb{N}}
\def\ZZ{\mathbb{Z}}
% Newcommand
\newcommand{\inner}[2]{\langle #1,#2\rangle} 
\newcommand{\numP}{\#\mathbf{P}} 
\renewcommand{\P}{\mathbf{P}}
\newcommand{\Var}[2][]{\mathbf{Var}_{#1}\left[#2\right]}
\newcommand{\E}[2][]{\mathbf{E}_{#1}\left[#2\right]}
\renewcommand{\emptyset}{\varnothing}
\newcommand{\ol}{\overline}
\newcommand{\argmin}{\mathop{\arg\min}}
\newcommand{\argmax}{\mathop{\arg\max}}
\renewcommand{\abs}[1]{\qty|#1|}
\newcommand{\defeq}{\triangleq} % triangle over =
\def\deq{\xlongequal{def}} % 'def' over =
\def\LHS{\text{LHS}}
\def\RHS{\text{RHS}}
\def\angbr#1{\langle#1\rangle} % <x>
\newcommand\set[1]{\qty{#1}}

\def\Esolve{\textcolor{blue}{Solve: }}
\def\Eproof{\textcolor{blue}{Proof: }}
\def\case#1{\textcolor{blue}{Case \uppercase\expandafter{\romannumeral#1}: }}
\def\card#1{\mbox{Card}(#1)}

% \newmdtheoremenv{lemma}{Lemma}
% \newmdtheoremenv{theorem}{\textcolor{red}{Theorem}}
% \newmdtheoremenv{defi}{\textcolor{blue}{Definition}}
\newtheorem{lemma}{Lemma}
\newtheorem{prp}{Proposition}
\newtheorem{thm}{Theorem}
\newtheorem{defi}{Defination}

\newenvironment{md}{\begin{mdframed}}{\end{mdframed}}
\usepackage{fancyhdr}
\pagestyle{fancy}
% \fancypagestyle{mainFancy}{
%     \fancyhf{}
%     \renewcommand\headrulewidth{.5pt}       % 页眉横线
%     \renewcommand\footrulewidth{0pt}
%     \fancyhead[OC]{\fzkai{\leftmark}}       % 页眉章标题
%     \fancyhead[EC]{\fzkai{\@title}}         % 页眉文章题目
% 	\lhead{\fzkai{author}}
%     \fancyhead[OR,EL]{\thepage}                 % 页眉编号
%     \fancyfoot[r]{\thumb} % 将拇指放到没有被使用的页眉或页脚处
% }
\fancyhf{}
\fancyhead[L]{\slshape{Haoyu Zhen}}
\fancyfoot[C]{\thepage}
\fancyhead[R]{\slshape{Available at \href{https://github.com/anyeZHY/AI2613-Homework}{https://github.com/anyeZHY/AI2613-Homework}}}

\graphicspath{{figures/}}

% \begin{document}
% \title{<++>}
\author{Haoyu Zhen}
% \maketitle
\setlength{\parindent}{0pt}
\setstretch{1.2}
% \end{document}


\begin{document}
\title{Homework 1}
\maketitle

\section{Probability Space of Tossing Coins}
\subsection*{\centering Problem 1}
Obviously, $\varnothing\notin\{C_s\}_{\{0,1\}^n}$.
\\
By the fact that $C_s\subset\Omega$ and  $\mbox{Card}\qty(\{0,1\}^n)<+\infty$, we have $\bigcup_{s\in\{0,1\}^n}C_s\subset\Omega$.
\\
Also, $\forall\omega\in\Omega$, $\omega\in C_{(\omega_1, \dots,\omega_n)}$, which means $\Omega\subset\bigcup_{s\in\{0,1\}^n}C_s$.
\\
$\forall s_1,s_2\in\set{0,1}^n$, $\forall\omega_1\in C_{s_1}$ and $\omega_2\in C_{s_2}$, $\omega_1\ne \omega_2$. Then $C_{s_1}\cap C_{s_2}\ne \emptyset$.
\\
Finally, $\set{C_s}_{s\in\set{0,1}^n}$ forms a partition of  $\Omega$.

\subsection*{\centering Problem 2}
Firstly, I will prove a lemma and a proposition.
\begin{defi}
	$\+S(\+C)\deq\set{\bigcup_{i=1, \ldots,m} A_i\mid m\in\set{1, \cdots,\abs{\+C}},A_i\in \+C}\cup\set{\emptyset}$
\end{defi}
\begin{lemma}
	\label{fc}
	If $\+C$ forms a partition of $\Omega$, then $\+S(\+C)$ is a  $\sigma$-algebra.
\end{lemma}
\begin{proof}
	$\emptyset,\Omega\in\+S(\+C)$.
	\\
	If $a\in\+S(\+C)$, then for some $B_i\in\+C$, $a=\bigcup_i B_i$. 
	Thus $\Omega-a=\bigcup_{A_i\in\+C} A_i-s\in\+S(\+C)$.
	\\
	If $A_i\in \+S(\+C)$, then $\bigcup_iA_i=\bigcup_i\qty(\bigcup_j B_{ij})\in\+S(\+C)$.
\end{proof}
\begin{defi}
	 $\+C_n={\set{C_s}_{s\in\set{0,1}^n}}$. 
\end{defi}
\begin{prp}
	$\+F_n=\+S({\+C_n})$.
\end{prp}
\begin{proof}
	Trivially, $\+C_n\subset\+S(\+C_n)$. Now we suppose that $\exists$ $\sigma$-algebra  $\+F$ s.t. $\+C_n\subset\+F$:
\\
For all $A\in\+S(\+C_n)$. Then $A=\bigcup C_s$ for some $s$ or $A=\emptyset$. By the defination of $\sigma$-algebra, $A\in\+F$, which means  $\+S(\+C_n)\in\+F$.
\\
Thus $\+S(\+C_n)$ is the minimal  $\sigma$-algebra containing sets in $\+C_n$.
\end{proof}
Now we have $\text{Card}(\+F_n)=\card{\+S(\+C_n)}=\text{Card}(2^{\+C_n})=\text{Card}(2^{\set{0,1}^n})=2^{2^n}$. We could say that $\+F_n\sim 2^{\set{0,1}^n}$ or $\+F_n$ and $2^{\set{0,1}^n}$ are \href{https://mathworld.wolfram.com/Equinumerous.html}{equinumerous}. So there exists a bijection between $\+F$ and  $2^{\set{0,1}^n}$.

\subsection*{\centering Problem 3}
By lemma \ref{fc}, $\forall n\in\NN$, $\+F_n=\+S(\+C_n)$.

And obviously $\+F_n\ne \+F_{n+1}$ because $\+C_n\ne \+C_{n+1}$ and $\text{Card}(\+C_n)\ne \text{Card}(\+C_{n+1})$.

$\forall A\in\+F_n$, $A=\bigcup_i C_{s_i}$ where $s_i\in\set{0,1}^n$. Then
\[
	A=
	\bigcup_i \qty(C_{[s_i,0]}\cup C_{[s_i,1]})
,\]
where the operator $[\cdot,\cdot]$ means a concatenation. For example, $[(0,0,1),1]=(0,0,1,1)$ and $[(0,0,1),(0,1)]=(0,0,1,0,1)$. I also use this operator in Problem 6 and 7.

Ultimately $\forall n\in\NN$, $\+F_n=\+S(\+C_n)\subsetneq \+S(\+C_{n+1})=\+F_{n+1}$.

\subsection*{\centering Problem 4}
By defination, $\+F_\infty=\bigcup_{n\ge 1}\+F_n$.

For all $A,B\in\+F_\infty$, $\exists m,n\in\NN$, $A\in\+F_m$ and $B\in\+F_n$. Let  $l\deq\max(m,n)$. Then  $A,B\in\+F_l$.

Thus $A^c,B^c\in\+F_l$ and  $A+B\in\+F_l\Longrightarrow A^c,B^c,A+B\in\+F_l$.

The proposition that $F_\infty$ is an algebra holds.

\begin{lemma}
	$\forall\omega\in\Omega$, $\forall n\in\NN$, $\set{\omega}\notin\+F_n$.
	\label{wf}
\end{lemma}
\begin{proof}
	$\forall A\in\+F_n$, $\card{A}=+\infty\ne \card{\set{\omega}}=1$
\end{proof}
Thus $2^{\Omega}\ne\+F_\infty$.


\subsection*{\centering Problem 5}
By lemma \ref{wf}, we have $\set{\omega}\ne\+F_\infty$.
% If not, there exists $n\in\NN$ such that  $\omega\notin\+F_n$, which lead to a contradiction.
Now we just need to prove that  $\set{\omega}\in\+B(\Omega)$.
% Intuitively, $\+B(\Omega)=2^\Omega$.

\textbf{Proof by contradiction.} Suppose that $\exists\omega'\notin\+B(\Omega)$.

$\forall n\in\NN$, $\+F_n\in\+B(\Omega)$.

A creative construction:
 \[
	 T\deq\bigcup_{n=1, \ldots, \infty}
	\set{\omega\mid\omega_1\ne\omega_1'\land\cdots\land\omega_n\ne\omega_n'}
	\deq\bigcup_{n=1, \ldots, \infty}T_n
.\] 

Here $T$ is a countable union and every $S_n$ is in  $\+S(\+C_n)=\+F_n$ ($T_n\in\+B(\Omega)$).
Hence $T\in\+B(\Omega)$. Intuitively,  $T=\Omega-{\omega}$.

Finnaly, $\set{\omega}=T^c\in\+B(\Omega)$.


\subsection*{\centering Problem 6}

For all $A\in\+F_\infty$,  $\exists m(A)\in\NN$ such that $A\in\+F_{m(A)}$ and  $\forall n\in\set{1,2, \cdots, m(A)-1}$, $A\notin\+F_n$. For simplicity, we denote  $m(A)$ as  $m$. It is worthy of mention that  $m$ \textbf{only} depends on  $A$.

Now we want to prove $\forall n\ge m$, $2{k_n}=k_{n+1}$.

If $A=C_{s_1}\cup\cdots\cup C_{s_{k_n}}$, then
\[
    A=
	\qty(C_{[s_1,0]}\cup C_{[s_1,1]})\cup\cdots\cup\qty(C_{[s_{k_n},0]}\cup C_{[s_{k_n},1]})
.\]

Trivially, $[s_1,0]\ne [s_1,1]\ne\cdots\ne[s_{n_k},0]\ne[s_{n_k},1]$. Thus we conclude $k_{n+1}=2k_n$ and  $\flatfrac{k}{2^n}=Const$ for any fixed  $A$.

\newpage
\subsection*{\centering Problem 7}
I've talked with Liyuan Mao and Wei Jiang for days. Here I will give several attempts I make, though I could not prove the proposition successfully.

\textcolor{blue}{\textbf{Attempt 1.}}
\begin{lemma}
	$\forall$ set $A$ and set $B$, $A\cap B=A-(A-B)=A-(A^c\cup B)^c=\qty[A^c\cup\qty(A^c\cup B)^c]^c$.
\end{lemma}
\begin{prp}
	$\+B(\Omega)=\set{\eval{\bigcup_{i=1}^\infty\qty(\bigcap_{j=1}^\infty A_{ij})}\, A_{ij}\in\+F_\infty}$.
\end{prp}
\begin{proof}
	Obviously, $\emptyset\in\+B(\Omega)$.
	
	$\forall A\in\+B(\Omega)$:
	$A^c=\qty[\bigcup_{i=1}^\infty\qty(\bigcap_{j=1}^\infty A_{ij})]^c=\bigcap_{i=1}^\infty\qty(\bigcup_{j=1}^\infty A_{ij}^c)$. (\textcolor{red}{Failure.} I could not prove this.)
\end{proof}

\textbf{Remark} of the \textbf{attempt 1}:
I want to construct $\+B$ directly, while I could not prove this set is a  $\sigma$-algebra. The idea is intuitively from  $\forall A\in \+B$, $A$ could be generated from  $C_{s_n}$ and  ${\omega_n}$.
\\
\\
\textcolor{blue}{\textbf{Attempt 2.}}
$\forall A\in\+B(\Omega)$:

$P_n(A)\deq P_n(\bigcup A_i)$ where $A_i\in F_n$ and $\bigcup A_i\mbox{ is the smallest cover of }A$. Then define $P(A)=\lim_{n\to\infty}P_n(A)$. (It is obvious that $P(A)$ is bounded and decreasing.)

\textcolor{red}{Hoever}, I could not prove the uniqueness of $P(A)$.
\\
\\
\textcolor{blue}{\textbf{Finally.}}

I find a more general theorem called Caratheodory's extension theorem. In this theorem we could find several creative constructions:
\[
	\lambda(E)\deq\inf\set{\sum_{j=1}^\infty\Pr(A_j):A_j\in\+F,E\subset\bigcup_j A_j}
\] and 
\[
	\bar{\+F}\deq\set{E:\lambda(F)=\lambda(F\cap E)+\lambda(F\cap E^c)\ \forall F\subset\Omega}
.\] 
\textbf{Final remark:} I found the similarity between the defination of $P_n(A)$ in Attempt 2 and  $\lambda(E)$.

Reference: \href{https://ocw.mit.edu/courses/electrical-engineering-and-computer-science/6-436j-fundamentals-of-probability-fall-2018/lecture-notes/MIT6_436JF18_lec02.pdf}{Fundamentals of Probability, MIT}.

\subsection*{\centering Problem 8}

Let $G_n=\set{\eval{\omega}\omega_1=\omega_2=\cdots=\omega_{n-1}=0,\omega_n=1 \mbox{ and } \omega\in\Omega}$ and $\Omega^*=\set{G_i}_{i\in\NN^+}$.

Then $X:\Omega^\mapsto \RR$, $X(\omega)=n$ such that $\omega\in G_n$.
Obviously $P(X=n)=\Pr(G_n)=\flatfrac{1}{2^n}$.


\section{Conditional Expectation}
\subsection*{\centering Problem 1}
$\forall A\in\+B$: $X^{-1}(A)=\set{\eval{\omega}X(\omega)\in A}\in\sigma(X)$ and $f^{-1}(A)\in\+B$.

Now we have $\+G\deq\set{\eval{X(\omega)}f(X(\omega))\in A}\in\+B$.
Thus 
\[
	\qty[f(X)]^{-1}(A)=\set{\eval{\omega}f(X(\omega))\in A}=\set{\eval{\omega}X(\omega)\in\+G}\in\sigma(X)
\]
which means that $f(X)$ is $\sigma(X)$-measurable.

\subsection*{\centering Problem 2}

\begin{prp}
	\label{sy}
	$\forall\omega\in\Omega$, $\exists A\in\sigma(Y)$ such that $Y^{-1}(\omega)=A$ and $\forall B\in\sigma(Y):A\cap B=\emptyset \lor A\subset B$.
\end{prp}
\begin{proof}
	If not, then $\exists B\ne\emptyset$, $A\cap B\ne A$. Let $A_1=A\cap B,A_2=A\cup B^c$. $A_1,A_2\in\sigma(Y)$ such that $A_1\cup A_2=Y^{-1}(\omega)$ and $A_1\cap A_2=\emptyset$. Obviously $\sigma(Y)$ is \textbf{not} the minimal $\sigma$-algebra such that Y is $\sigma(Y)$-measurable. Contradiction.
\end{proof}

Then $\forall \omega\in\Omega$, $Y^{-1}\qty[Y(\omega)]=Y'^{-1}\qty[Y'(\omega)]$. (If not, let  $B=Y^{-1}\qty[Y(\omega)]\cap Y'^{-1}\qty[Y'(\omega)]$. Then $B$ is a set which contradicts the Proposition \ref{sy}.


\subsection*{\centering Problem 3}

$\expect[X\mid\+F]=\expect[X\mid Y]$ where $\sigma(Y)=\+F$.

\textbf{Remark:} after I finnished Problem 3 and 4 I found that the existence of $Y$ may not always hold for every $\+F$. The existence of $Y$ depends on whether  $\+F$ has partitions, (i.e., if we could find the partitions of $\+F$, then we have a  $Y$). Hoever, for more general cases I could not define it clealy (may use Lebergue Integral).
\subsection*{\centering Problem 4}

Suppose that  $\sigma(Y_1)=\+F_1$ and $\sigma(Y_2)=\+F_2$.

\begin{equation}
	\begin{aligned}
		\expect[\expect[ X\mid\+F_1]\mid\+F_2](\omega_0)
		&=\sum_{a\in\text{Ran}(f_Y)}a\times\Pr[f_{Y_1}=a\mid Y_2=Y_2(\omega_0)]
		\\&=
		f_{Y_1}(\omega_0)\times\Pr[f_{Y_1}=f_{Y_1}(\omega_0)\mid Y_2=Y_2(\omega_0)]
		\\&=
		f_{Y_1}(\omega_0)
	\end{aligned}
	\label{e_1}
\end{equation}

The second euqality of Eq.\ref{e_1} holds by the fact that:
\[
	\forall\omega\mbox{ such that }a=f_{Y_1}(\omega)\ne f_{Y_1}(\omega_0)\mbox{, }\qty[f_{Y_1}=a]\cap\qty[Y_2=Y_2(\omega_0)]=\emptyset
.\]

The third euqality of Eq.\ref{e_1} holds by the fact that:
\[
	[Y_2=Y_2(\omega_0)]\subset[f_{Y_1}=f_{Y_1}(\omega_0)]
.\] 

Let $\+D_0\deq \set{b\in\RR\mid \exists \omega\in\Omega \mbox{ such that } Y_2^{-1}(Y_2(\omega))\subset Y_1^{-1}(Y_1(\omega))\mbox{ and } b=f_{Y_2}(\omega)}$. The interpretation of $\+D_0$: $\forall A\in\sigma(Y_1)$, there exsists serveral sets $A_i\in\sigma(Y_2)$ such that $\bigcup A_i=A$. Here we let $A=Y_1^{-1}(\omega)$ and $\set{A_i}$ is the partition of $\sigma(Y_2)$.

\begin{equation}
	\begin{aligned}
		\expect[\expect[X\mid Y_2]\mid Y_1](\omega_0)
		&=
		\frac{\sum_{b\in\mathrm{Ran}(f_{Y_2})}b\times\Pr(f_{Y_2}=b\cap Y_1=Y_1(\omega_0))}
		{\Pr(Y_1=Y_1(\omega))}
		\\[5pt]&=
		\frac{\sum_{b\in\+D_0}b\times\Pr(f_{Y_2}=b\cap Y_1=Y_1(\omega_0))}
		{\Pr(Y_1=Y_1(\omega))}
		\\[5pt]&=
		\frac{\sum_{b\in\+D_0}\expect(X\mid Y_2=Y_2(\omega_b))\Pr(f_{Y_2}=b)}
		{\Pr(Y_1=Y_1(\omega))}
		\\[5pt]&=
		\frac{\sum_{b\in\+D_0}\expect(X\mid Y_2=Y_2(\omega_b))\Pr({Y_2}=Y_2(\omega_b))}
		{\Pr(Y_1=Y_1(\omega))}
		\\[5pt]&=
		\frac{\sum_{b\in\+D_0}\sum_a\expect(X=a\mid Y_2\cap Y_2(\omega_b))}
		{\Pr(Y_1=Y_1(\omega))}
		\\[5pt]&=
		\frac{\sum_a\sum_{b\in\+D_0}\expect(X=a\mid Y_2\cap Y_2(\omega_b))}
		{\Pr(Y_1=Y_1(\omega))}
		\\[5pt]&=
		\frac{\sum_a\Pr(X=a\cap Y_1=Y_1(\omega_0))}
		{\Pr(Y_1=Y_1(\omega))}
		\\[5pt]&=
		f_{Y_1}(\omega_0)
	\end{aligned}
	\label{e_2}
\end{equation}

Ultimately, with EQ. \ref{e_1} and EQ. \ref{e_2} we prove the proposition of problem 4.


\end{document}

