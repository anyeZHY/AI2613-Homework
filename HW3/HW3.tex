% ! Tex program = xelatex
\documentclass{article}
% 中文
% \usepackage[UTF8]{ctex}

% For more choices
% %! Tex program = xelatex
% \documentclass{article}
%中文
%\usepackage[UTF8]{ctex}
%数学公式
\usepackage{amsmath,amssymb}
%\usepackage{ntheorem}
% \usepackage[framemethod=TikZ]{mdframed}
\usepackage{amsthm}
%边界
\usepackage[letterpaper,top=2.5cm,bottom=2.5cm,left=3cm,right=3cm,marginparwidth=1.75cm]{geometry}%table package
%Table
\usepackage{multirow,booktabs}
\usepackage{makecell}
%字体颜色
\usepackage{color}
\usepackage[dvipsnames]{xcolor}  % 更全的色系
%代码
\usepackage[OT1]{fontenc}
% MATLAB 代码风格
%\usepackage[framed,numbered,autolinebreaks,useliterate]{/Users/anye_zhenhaoyu/Desktop/Latex/mcode}
\usepackage{listings}
\usepackage{algorithm}
\usepackage{algorithmic}
\usepackage{pythonhighlight} % Python
%插图
\usepackage{graphicx}
%改变item格式
\usepackage{enumerate}
%物理
\usepackage{physics}
%extra arrows
\usepackage{extarrows}
% caption(居中指令)
%\usepackage[justification=centering]{caption}
\usepackage{caption}
% htpb
\usepackage{stfloats}
% pdf 拼接
\usepackage{pdfpages}
% 超链接url
\usepackage{url}
% \usepackage{tikz}
\usepackage{pgfplots}
\pgfplotsset{compat=newest}
\usepackage[colorlinks=true, allcolors=blue]{hyperref}
\usepackage{setspace}

% --------------definations-------------- %
\def\*#1{\boldsymbol{#1}}
\def\+#1{\mathcal{#1}} 
\def\-#1{\bar{#1}}
% Domains
\def\expect{\mathbb{E}}
\def\RR{\mathbb{R}}
\def\CC{\mathbb{C}}
\def\NN{\mathbb{N}}
\def\ZZ{\mathbb{Z}}
% Newcommand
\newcommand{\inner}[2]{\langle #1,#2\rangle} 
\newcommand{\numP}{\#\mathbf{P}} 
\renewcommand{\P}{\mathbf{P}}
\newcommand{\Var}[2][]{\mathbf{Var}_{#1}\left[#2\right]}
\newcommand{\E}[2][]{\mathbf{E}_{#1}\left[#2\right]}
\renewcommand{\emptyset}{\varnothing}
\newcommand{\ol}{\overline}
\newcommand{\argmin}{\mathop{\arg\min}}
\newcommand{\argmax}{\mathop{\arg\max}}
\renewcommand{\abs}[1]{\qty|#1|}
\newcommand{\defeq}{\triangleq} % triangle over =
\def\deq{\xlongequal{def}} % 'def' over =
\def\LHS{\text{LHS}}
\def\RHS{\text{RHS}}
\def\angbr#1{\langle#1\rangle} % <x>
\newcommand\set[1]{\qty{#1}}

\def\Esolve{\textcolor{blue}{Solve: }}
\def\Eproof{\textcolor{blue}{Proof: }}
\def\case#1{\textcolor{blue}{Case \uppercase\expandafter{\romannumeral#1}: }}
\def\card#1{\mbox{Card}(#1)}

% \newmdtheoremenv{lemma}{Lemma}
% \newmdtheoremenv{theorem}{\textcolor{red}{Theorem}}
% \newmdtheoremenv{defi}{\textcolor{blue}{Definition}}
\newtheorem{lemma}{Lemma}
\newtheorem{prp}{Proposition}
\newtheorem{thm}{Theorem}
\newtheorem{defi}{Defination}

\newenvironment{md}{\begin{mdframed}}{\end{mdframed}}
\usepackage{fancyhdr}
\pagestyle{fancy}
% \fancypagestyle{mainFancy}{
%     \fancyhf{}
%     \renewcommand\headrulewidth{.5pt}       % 页眉横线
%     \renewcommand\footrulewidth{0pt}
%     \fancyhead[OC]{\fzkai{\leftmark}}       % 页眉章标题
%     \fancyhead[EC]{\fzkai{\@title}}         % 页眉文章题目
% 	\lhead{\fzkai{author}}
%     \fancyhead[OR,EL]{\thepage}                 % 页眉编号
%     \fancyfoot[r]{\thumb} % 将拇指放到没有被使用的页眉或页脚处
% }
\fancyhf{}
\fancyhead[L]{\slshape{Haoyu Zhen}}
\fancyfoot[C]{\thepage}
\fancyhead[R]{\slshape{Available at \href{https://github.com/anyeZHY/AI2613-Homework}{https://github.com/anyeZHY/AI2613-Homework}}}

\graphicspath{{figures/}}

% \begin{document}
% \title{<++>}
\author{Haoyu Zhen}
% \maketitle
\setlength{\parindent}{0pt}
\setstretch{1.2}
% \end{document}


% \input{/Users/anye_zhenhaoyu/Desktop/ln_preamble.tex}

% %! Tex program = xelatex
% \documentclass{article}
%中文
%\usepackage[UTF8]{ctex}
%数学公式
\usepackage{amsmath,amssymb}
%\usepackage{ntheorem}
% \usepackage[framemethod=TikZ]{mdframed}
\usepackage{amsthm}
%边界
\usepackage[letterpaper,top=2.5cm,bottom=2.5cm,left=3cm,right=3cm,marginparwidth=1.75cm]{geometry}%table package
%Table
\usepackage{multirow,booktabs}
\usepackage{makecell}
%字体颜色
\usepackage{color}
\usepackage[dvipsnames]{xcolor}  % 更全的色系
%代码
\usepackage[OT1]{fontenc}
% MATLAB 代码风格
%\usepackage[framed,numbered,autolinebreaks,useliterate]{/Users/anye_zhenhaoyu/Desktop/Latex/mcode}
\usepackage{listings}
\usepackage{algorithm}
\usepackage{algorithmic}
\usepackage{pythonhighlight} % Python
%插图
\usepackage{graphicx}
%改变item格式
\usepackage{enumerate}
%物理
\usepackage{physics}
%extra arrows
\usepackage{extarrows}
% caption(居中指令)
%\usepackage[justification=centering]{caption}
\usepackage{caption}
% htpb
\usepackage{stfloats}
% pdf 拼接
\usepackage{pdfpages}
% 超链接url
\usepackage{url}
% \usepackage{tikz}
\usepackage{pgfplots}
\pgfplotsset{compat=newest}
\usepackage[colorlinks=true, allcolors=blue]{hyperref}
\usepackage{setspace}

% --------------definations-------------- %
\def\*#1{\boldsymbol{#1}}
\def\+#1{\mathcal{#1}} 
\def\-#1{\bar{#1}}
% Domains
\def\expect{\mathbb{E}}
\def\RR{\mathbb{R}}
\def\CC{\mathbb{C}}
\def\NN{\mathbb{N}}
\def\ZZ{\mathbb{Z}}
% Newcommand
\newcommand{\inner}[2]{\langle #1,#2\rangle} 
\newcommand{\numP}{\#\mathbf{P}} 
\renewcommand{\P}{\mathbf{P}}
\newcommand{\Var}[2][]{\mathbf{Var}_{#1}\left[#2\right]}
\newcommand{\E}[2][]{\mathbf{E}_{#1}\left[#2\right]}
\renewcommand{\emptyset}{\varnothing}
\newcommand{\ol}{\overline}
\newcommand{\argmin}{\mathop{\arg\min}}
\newcommand{\argmax}{\mathop{\arg\max}}
\renewcommand{\abs}[1]{\qty|#1|}
\newcommand{\defeq}{\triangleq} % triangle over =
\def\deq{\xlongequal{def}} % 'def' over =
\def\LHS{\text{LHS}}
\def\RHS{\text{RHS}}
\def\angbr#1{\langle#1\rangle} % <x>
\newcommand\set[1]{\qty{#1}}

\def\Esolve{\textcolor{blue}{Solve: }}
\def\Eproof{\textcolor{blue}{Proof: }}
\def\case#1{\textcolor{blue}{Case \uppercase\expandafter{\romannumeral#1}: }}
\def\card#1{\mbox{Card}(#1)}

% \newmdtheoremenv{lemma}{Lemma}
% \newmdtheoremenv{theorem}{\textcolor{red}{Theorem}}
% \newmdtheoremenv{defi}{\textcolor{blue}{Definition}}
\newtheorem{lemma}{Lemma}
\newtheorem{prp}{Proposition}
\newtheorem{thm}{Theorem}
\newtheorem{defi}{Defination}

\newenvironment{md}{\begin{mdframed}}{\end{mdframed}}
\usepackage{fancyhdr}
\pagestyle{fancy}
% \fancypagestyle{mainFancy}{
%     \fancyhf{}
%     \renewcommand\headrulewidth{.5pt}       % 页眉横线
%     \renewcommand\footrulewidth{0pt}
%     \fancyhead[OC]{\fzkai{\leftmark}}       % 页眉章标题
%     \fancyhead[EC]{\fzkai{\@title}}         % 页眉文章题目
% 	\lhead{\fzkai{author}}
%     \fancyhead[OR,EL]{\thepage}                 % 页眉编号
%     \fancyfoot[r]{\thumb} % 将拇指放到没有被使用的页眉或页脚处
% }
\fancyhf{}
\fancyhead[L]{\slshape{Haoyu Zhen}}
\fancyfoot[C]{\thepage}
\fancyhead[R]{\slshape{Available at \href{https://github.com/anyeZHY/AI2613-Homework}{https://github.com/anyeZHY/AI2613-Homework}}}

\graphicspath{{figures/}}

% \begin{document}
% \title{<++>}
\author{Haoyu Zhen}
% \maketitle
\setlength{\parindent}{0pt}
\setstretch{1.2}
% \end{document}


% \input{/path/ln_preamble.tex}

% On my MAC's Desktop
%! Tex program = xelatex
% \documentclass{article}
%中文
%\usepackage[UTF8]{ctex}
%数学公式
\usepackage{amsmath,amssymb}
%\usepackage{ntheorem}
% \usepackage[framemethod=TikZ]{mdframed}
\usepackage{amsthm}
%边界
\usepackage[letterpaper,top=2.5cm,bottom=2.5cm,left=3cm,right=3cm,marginparwidth=1.75cm]{geometry}%table package
%Table
\usepackage{multirow,booktabs}
\usepackage{makecell}
%字体颜色
\usepackage{color}
\usepackage[dvipsnames]{xcolor}  % 更全的色系
%代码
\usepackage[OT1]{fontenc}
% MATLAB 代码风格
%\usepackage[framed,numbered,autolinebreaks,useliterate]{/Users/anye_zhenhaoyu/Desktop/Latex/mcode}
\usepackage{listings}
\usepackage{algorithm}
\usepackage{algorithmic}
\usepackage{pythonhighlight} % Python
%插图
\usepackage{graphicx}
%改变item格式
\usepackage{enumerate}
%物理
\usepackage{physics}
%extra arrows
\usepackage{extarrows}
% caption(居中指令)
%\usepackage[justification=centering]{caption}
\usepackage{caption}
% htpb
\usepackage{stfloats}
% pdf 拼接
\usepackage{pdfpages}
% 超链接url
\usepackage{url}
% \usepackage{tikz}
\usepackage{pgfplots}
\pgfplotsset{compat=newest}
\usepackage[colorlinks=true, allcolors=blue]{hyperref}
\usepackage{setspace}

% --------------definations-------------- %
\def\*#1{\boldsymbol{#1}}
\def\+#1{\mathcal{#1}} 
\def\-#1{\bar{#1}}
% Domains
\def\expect{\mathbb{E}}
\def\RR{\mathbb{R}}
\def\CC{\mathbb{C}}
\def\NN{\mathbb{N}}
\def\ZZ{\mathbb{Z}}
% Newcommand
\newcommand{\inner}[2]{\langle #1,#2\rangle} 
\newcommand{\numP}{\#\mathbf{P}} 
\renewcommand{\P}{\mathbf{P}}
\newcommand{\Var}[2][]{\mathbf{Var}_{#1}\left[#2\right]}
\newcommand{\E}[2][]{\mathbf{E}_{#1}\left[#2\right]}
\renewcommand{\emptyset}{\varnothing}
\newcommand{\ol}{\overline}
\newcommand{\argmin}{\mathop{\arg\min}}
\newcommand{\argmax}{\mathop{\arg\max}}
\renewcommand{\abs}[1]{\qty|#1|}
\newcommand{\defeq}{\triangleq} % triangle over =
\def\deq{\xlongequal{def}} % 'def' over =
\def\LHS{\text{LHS}}
\def\RHS{\text{RHS}}
\def\angbr#1{\langle#1\rangle} % <x>
\newcommand\set[1]{\qty{#1}}

\def\Esolve{\textcolor{blue}{Solve: }}
\def\Eproof{\textcolor{blue}{Proof: }}
\def\case#1{\textcolor{blue}{Case \uppercase\expandafter{\romannumeral#1}: }}
\def\card#1{\mbox{Card}(#1)}

% \newmdtheoremenv{lemma}{Lemma}
% \newmdtheoremenv{theorem}{\textcolor{red}{Theorem}}
% \newmdtheoremenv{defi}{\textcolor{blue}{Definition}}
\newtheorem{lemma}{Lemma}
\newtheorem{prp}{Proposition}
\newtheorem{thm}{Theorem}
\newtheorem{defi}{Defination}

\newenvironment{md}{\begin{mdframed}}{\end{mdframed}}
\usepackage{fancyhdr}
\pagestyle{fancy}
% \fancypagestyle{mainFancy}{
%     \fancyhf{}
%     \renewcommand\headrulewidth{.5pt}       % 页眉横线
%     \renewcommand\footrulewidth{0pt}
%     \fancyhead[OC]{\fzkai{\leftmark}}       % 页眉章标题
%     \fancyhead[EC]{\fzkai{\@title}}         % 页眉文章题目
% 	\lhead{\fzkai{author}}
%     \fancyhead[OR,EL]{\thepage}                 % 页眉编号
%     \fancyfoot[r]{\thumb} % 将拇指放到没有被使用的页眉或页脚处
% }
\fancyhf{}
\fancyhead[L]{\slshape{Haoyu Zhen}}
\fancyfoot[C]{\thepage}
\fancyhead[R]{\slshape{Available at \href{https://github.com/anyeZHY/AI2613-Homework}{https://github.com/anyeZHY/AI2613-Homework}}}

\graphicspath{{figures/}}

% \begin{document}
% \title{<++>}
\author{Haoyu Zhen}
% \maketitle
\setlength{\parindent}{0pt}
\setstretch{1.2}
% \end{document}



\graphicspath{{figures/}}

\begin{document}
% \tableofcontents
\title{\vspace{-1.5cm}Homework 3}
\maketitle
\section*{Problem 1 (FTMC for countably infinite chains)}

\textbf{(1).}
We just need to prove [F]+[I]$\implies$[PR].

By [I] and [F], $\forall i\in[n]$, $\exists t_1,t_2,\alpha,\beta$ such that: $\P_i[X_{t_1}=i]=\alpha$ and $\forall j\ne i$, $\P_j[X_{t_2}=i]>\beta$.

Then we have $\exists t$, $\P_i[T_i\le t]>\alpha$ and $\P_j[T_i\le t]>\beta$. Thus
\[
	\P_i[T_i<nt] = 1-(1-\alpha)\beta(1-\beta)^n \implies \P_i[T_i<\infty] = 1
\] and
\[
	\=E[T_i]<\alpha(1+\cdots+t)+\sum_{n=0}^\infty\beta(1-\alpha)(1-\beta)^n\qty[\frac{t(t+1)}{2}+(n-1)t^2]<\infty
	.\] Nota Bene: I use $\sum_{n=0}^\infty n\alpha^n<\infty$ where $\abs{\alpha}<1$.
Finally,
\[
	\text{[F]+[A]+[I]}\implies\text{[PR]+[A]+[I]}\implies\text{[S]+[U]+[C]}
	.\]
\textbf{(2).} Let $Q$ be the transition function of the markov chain on $\Omega^2$. $Q$ has several properties:
\begin{itemize}
	\item
	      $Q$ has a stationary  $\pi_{(i,j)}=\pi_i\pi_j$ by the fact that
	      \[
		      \pi_i\pi_j
		      =
		      \sum_kP(k,i)\pi_k\times\sum_lP(l,j)\pi_l
		      =
		      \sum_k\sum_lP(k,i)P(l,j)\pi_k\pi_l
		      .\]
	\item
	      $Q$ is irreducible: for all $i,j,k,l\in\NN$, there exists $t\in\NN$ such that
	      \[
		      Q^t[(i,j),(k,l)]=P^t(i,k)P^t(j,l)>0
		      .\]
\end{itemize}
Then we have $\pi_{(i,j)}=\flatfrac{1}{\=E_{(i,j)}[T_{(i,j)}]>0}$, which means that $P_{(i,j)}[T_{(i,j)}<\infty]=1$. This could entail $P_{(i,j)}[T_{(k,k)}<\infty]=1$.
\\\\
\textbf{(3).}
We have proved [PR]+[I]$\implies$[S]+[U]. Now we construct a coupling and update $X^t$ and  $Y^t$ by following rules: ($Y^0=\pi$ and $X^0=\mu$)
\[
	\begin{cases}
		X^{t+1}=X^t,\,Y^{t+1}=Y^t                                & \text{if }X^t=Y^t
		\\
		X^{t}\to X^{t+1},\,Y^{t}\to Y^{t+1}\text{ Independently} & \text{if }X^t\ne Y^t
	\end{cases}
	.\]
By what we've proved in (2), $\forall i,j$, $\exists T$, $(X^t=i,Y^t=j)$ and $X^{t+T}=Y^{t+T}=k$. Thus we have $D_{TV}(\mu,\pi)\le \lim_{t\to\infty}\Pr(X^t\ne Y^t)=0$.
Then [C] holds for countably infinite chain.

\section*{Problem 2 (A Randomized Algorithm for 3-SAT)}
\textbf{(1).}
Fistly we have $\P_{2n^2}\defeq\Pr\qty(\exists t\in[0,2n^2],\,\-{s.t.}Y_t=n)\ge \flatfrac{1}{2}$. Then
\[
	\P[\text{the new ALGO outputs the correct answer}]
	=
	1-(1-\P_{2n^2})^{50}\le 1-0.5^{50}
	.\]
\textbf{(2).}
Suppose the clause is $p\lor q\lor r$. Let $\sigma(a)$ denote the ground truth of $a$. If $\sum_{p,q,r}1[\sigma(i)=\-{True}]=i$, then $\Pr[X_{t+1}=X_t+1]=\flatfrac{i}{3}$.
Thus
\[
	\Pr[X_{t+1}=X_t+1]\ge \frac{1}{3}\text{ and }\Pr[X_{t+1}=X_t-1]\le\frac{2}{3}
	.\]
\textbf{(3).}
Consider the 1-D randowalk $\{Y_t\}$:
\[
	Y_{t+1}=
	\begin{cases}
		Y_t+1 & \text{w.p. }\flatfrac{1}{3} \\
		Y_t-1 & \text{w.p. }\flatfrac{2}{3} \\
	\end{cases}
\] where $Y_t\ne 0$ or $n$.
If $Y_t=0$, $Y_{t+1}=Y_t+1$ w.p. 1 and if $Y_t=n$,then $Y_{t+1}=Y_t-1$ w.p. 1. Then
\[
	\=E[T_{i\to i+1}]=\frac{1}{3}+\frac{2}{3}(1+\=E[T_{i-1\to i}]+\=E[T_{i\to i+1}])
	,\] which entails
\[
	\=E[T_{i\to i+1}]=2\=E[T_{i-1\to i}]+3
	.\] By $\=E[t_{0\to 1}]=1$:
\[
	\=E[T_{i\to i+1}]=2^{i+2}-3
	.\] Thus
\[
	\=E[T_{i\to n}]=\sum_{k=i}^{n-1}\=E[T_{k\to k+1}]=2^{n+2}-2^i-3(n-i)<2^{n+2}
	.\]
Finally,
\[
	1-\Pr\qty(\exists t\in[0,400\cdot2^n]:Y_t=n)
	=
	\Pr[T_{Y_0\to n}>400\cdot2^n]
	\le
	\frac{\=E[T_{Y_0\to n}]}{400\cdot2^n}
	=
	0.01
	.\]
\textbf{(4).} $Y_0=n-i$.
\[
	\begin{aligned}
		\Pr\qty(\exists t\in[0,3n]:Y_t=n)
		\ge&
		\Pr\qty(Y_{3i}=n)
		\\=&
		\binom{3i}{i}\qty(\frac{1}{3})^{2i}\qty(\frac{2}{3})^i
		\\[5pt]\ge &
		\sqrt{\frac{3}{4\pi i}}\qty(\frac{27}{4})^i\qty(\frac{1}{3})^{2i}\qty(\frac{2}{3})^i
		\\=&
		\sqrt{\frac{3}{4\pi i}}\qty(\frac{1}{2})^i
	\end{aligned}
.\]
The second inequality holds by stiling formula.

\newpage
\textbf{(5).} By some tricks of inequality:
\[
	\begin{aligned}
		&\Pr[\text{Output a satisfying assignment}]
		\\=&
		\frac{1}{2^n}\sum_{i=0}^n \binom{n}{i}\sqrt{\frac{3}{4\pi i}}\qty(\frac{1}{2})^i
		\\\ge &
		\sqrt{\frac{3}{4\pi n}}\frac{1}{2^n}\sum_{i=0}^n \binom{n}{i}\qty(\frac{1}{2})^i
		\\=&
		\sqrt{\frac{3}{4\pi n}}\frac{1}{2^n}\qty(\frac{3}{2})^n		
		\\=&
		\sqrt{\frac{3}{4\pi n}}\qty(\frac{3}{4})^n		
	\end{aligned}
.\] 
\textbf{(6).}
In order to:
\[
	\qty[1-\sqrt{\frac{3}{4\pi n}}\qty(\frac{3}{4})^n]^k
	\le 0.01
.\] 
$k$ should satisfy:
\[
	k\ge -\flatfrac{2}{\log_{10}\qty[1-\sqrt{\frac{3}{4\pi n}}\qty(\frac{3}{4})^n]}
.\] By the fact that $-\ln(1-x)>x$,
\[
	k=\sqrt{\frac{16\pi}{3}}\ln(10)\times
	\sqrt{n}\qty(\frac{4}{3})^n
	=
	\+O\qty[n^{0.5}\qty(\frac{4}{3})^n]
.\] 
The algoithm should be:

Repeat for $k$ times: ``repeat the flipping process for $3n$ times, starting with some $\sigma_0$ which is uniform at random from all $2n$ assignments of the variables''.

The complexity is \[\+O(nk)=\+O\qty[n^{1.5}\qty(\frac{4}{3})^n].\]
\\[5pt]
Reference of problem 2: \href{http://www.cs.nthu.edu.tw/~wkhon/random12/lecture/lecture22.pdf}{Randomized Algorithms, National Tsing Hua University}.
\end{document}
